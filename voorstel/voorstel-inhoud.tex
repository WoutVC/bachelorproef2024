%%---------- Inleiding ---------------------------------------------------------

\section{Introductie}%
\label{sec:introductie}

Deze bachelorproef richt zich op het onderzoek en begrijpen van de complexiteit rond gegevensopslag in Edge-omgevingen,
 met een specifieke focus op Edge Database-architecturen. \\
 
Het doel van dit onderzoek is om niet alleen praktische aanbevelingen te formuleren maar eerder het verkennen van
nieuwe inzichten en kennis op het gebied van gegevensopslag in \\ Edge-omgevingen. \\

We gaan niet enkel de bestaande informatie over dit onderwerp verzamelen, maar ook diepgaand onderzoek doen
 om nieuwe inzichten en oplossingen te bekomen. \\

De beoogde doelgroep van deze studie zijn werknemers die regelmatig werken met gegevensbeheer in Edge-omgevingen. \\

Het onderzoek is nauwkeurig afgestemd op een specifieke probleemsituatie binnen het vakgebied,
 gericht op het begrijpen en verbeteren van Edge Database-architecturen. \\

De kernprobleemstelling concentreert zich op het optimaliseren van gegevensopslag in Edge-omgevingen,
 met als centrale onderzoeksvraag: 
 "Hoe kunnen Edge Database-architecturen geoptimaliseerd worden om efficiënte gegevensopslag in Edge-omgevingen te faciliteren?" \\

Dit onderzoek bestudeert niet enkel de behoefte van Edge Database-architecturen,
maar ook de rol van verschillende technologieën
  in het optimaliseren van gegevensopslag in Edge-omgevingen. \\ \\

Dit vereist een uitgebreide analyse van de huidige stand van zaken en het verkennen van innovatieve oplossingen.

%---------- State of the art ---------------------------------------------------

\section{Stand van zaken}%
\label{sec:state-of-the-art}

Deze sectie onderzoekt de huidige stand van zaken met betrekking tot Database on Edge. \\

Hierbij ligt de focus op het begrijpen van de essentie, verschillen met traditionele databases, voordelen, uitdagingen,
 en optimalisatiemogelijkheden.

%---------- Database on Edge ---------------------------------------------------
\subsection{Database on Edge}%
\label{subsec:database_on_edge}

Database on Edge is een opkomende benadering binnen database technologie,
 voortkomend uit het gebruik van Edge computing  \cite{Yang2019EdgeDBAE}.  \\
 
In tegenstelling tot traditionele databases worden Edge databases verspreid over verschillende lokale apparaten,
 waardoor gegevens lokaal verwerkt en opgeslagen kunnen worden, in plaats van centraal in de cloud. \\

Deze innovatieve benadering richt zich, vergelijkbaar met moderne bedrijven die streven naar efficiëntie en innovatie,
 op het verminderen van latency en het verbeteren van de prestaties. \\
 
Het concept is ontstaan uit de noodzaak om real-time interacties mogelijk te maken, wat cruciaal is in diverse toepassingen
\cite{Yang2019EdgeDBAE}. \\

Voordelen van Database on Edge omvatten lokale gegevensverwerking, verminderde latency en verbeterde prestaties,
 waardoor het geschikt is voor toepassingen waar snelle respons cruciaal is.

Echter, zoals bij elke technologische benadering, zijn er uitdagingen verbonden aan Database on Edge,
 waaronder het beheer van diverse databases over verschillende apparaten,
 \\ optimalisatie van resourcegebruik op Edge Devices, en het waarborgen van consistente prestaties. \\

De optimalisatie van Database on Edge is een actief onderzoeksgebied 
 en omvat het verkennen van efficiënte algoritmen voor gegevensverwerking,
  verbeteringen in gegevensopslag, en het beheer van gedistribueerde databases om de prestaties te maximaliseren.

\newpage

%---------- Methodologie ------------------------------------------------------

\section{Methodologie}%
\label{sec:methodologie}

Om de onderzoeksvraag te beantwoorden, gaan we gebruik maken van een combinatie van literatuurstudie,
 interviews met relevante bedrijven en een grondige vergelijkende studie. \\
 
Specifieke methoden, zoals requirements-analyse en experimenten, worden toegepast.

We gaan geen enquetes uitvoeren, omdat we de voorkeur geven aan een meer persoonlijke aanpak. \\
 
De technische diepte van deze bachelorproef wordt benadrukt,
 inclusief de beschrijving van gebruikte tools zoals hardware, software en diensten. \\

Een gedetailleerde tijdsplanning wordt opgesteld om de duur van elke onderzoeksfase en de concrete deliverables te definiëren.
Deze tijdsplanning wordt opgesteld in de vorm van een Gantt chart. Deze zal u vinden op de laatste pagina van dit voorstel.




%---------- Verwachte resultaten ----------------------------------------------

\section{Verwachte resultaten en Conclusie}%
\label{sec:verwachte_resultaten}

De verwachte resultaten omvatten concrete aanbevelingen voor werknemers met betrekking tot de optimalisatie
van gegevensopslag in Edge-omgevingen. \\

Grafieken met verwachte conclusies zullen worden gepresenteerd,
 de eerste grafiek zal gaan over het vergelijken van ongeoptimaliseerde en geoptimaliseerde systemen. \\

Met als x-as de tijd en op de y-as de latency per milliseconde. \\

In de conclusie wordt benadrukt hoe de verwachte resultaten bijdragen aan het oplossen van de gedefinieerde probleemsituatie. \\

De conclusie zal gebaseerd zijn op de verzamelde gegevens en resultaten van het onderzoek.

\newpage
\section{Tijdsplanning Methodologie}%
\label{sec:tijdsplanning}
\begin{ganttchart}[
  hgrid,
  vgrid={*{6}{draw=none},dotted},
  x unit=0.14cm,
  y unit title=0.6cm,
  y unit chart=0.6cm,
  title height=0.5,
  bar height=0.5,
  bar/.append style={
      fill=blue!50,
  },
  time slot format=isodate,
  time slot unit=day,
  link bulge=5,
  bar label node/.style={font=\tiny},
  calendar week text={W\currentweek}
]{2024-02-01}{2024-05-30}

\gantttitlecalendar{year, month=shortname, week} \\

\ganttbar{Indienen verbeterd voorstel}{2024-02-16}{2024-02-16} \\
\ganttlinkedbar{Bronnen zoeken}{2024-02-17}{2024-02-20} \\
\ganttlinkedbar{Bronnen uitschrijven}{2024-02-21}{2024-03-25} \\
\ganttlinkedbar{Analyse Methodologie}{2024-03-26}{2024-04-05} \\
\ganttlinkedbar{Conclusie Methodologie}{2024-04-06}{2024-05-20} \\
\ganttlinkedbar{Bachelorproef herlezen}{2024-05-21}{2024-05-25} \\
\ganttlinkedbar{Bachelorproef indienen}{2024-05-25}{2024-05-26} \\

\end{ganttchart}

%```mermaid
%flowchart TD
    %A[Indienen verbeterd voorstel] --> B[Bronnen zoeken];
    %B --> C[Bronnen uitschrijven];
    %C --> D[Analyse Methodologie];
    %D --> E[Conclusie Methodologie];
    %E --> F[Bachelorproef herlezen];
    %F --> G[Bachelorproef indienen];
%```
