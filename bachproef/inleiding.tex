\chapter{\IfLanguageName{dutch}{Inleiding}{Introduction}}%
\label{ch:inleiding}

In deze inleiding geef ik een overzicht van het onderwerp van mijn bachelorproef en waarom het de moeite waard is om te onderzoeken. Ik zal de context en achtergrond van het onderwerp schetsen, de probleemstelling en onderzoeksvraag uiteenzetten, en de opzet van deze bachelorproef bespreken.

\section{\IfLanguageName{dutch}{Probleemstelling}{Problem Statement}}%
\label{sec:probleemstelling}

Het onderzoek richt zich op het begrijpen van de complexiteit rond gegevensopslag in Edge-omgevingen, met als specifieke focus Edge Database-architecturen. De uitdagingen en mogelijkheden van deze architecturen hebben directe relevantie voor bedrijven die streven naar efficiënte gegevensopslag in moderne IT-infrastructuren.

\section{\IfLanguageName{dutch}{Onderzoeksvraag}{Research question}}%
\label{sec:onderzoeksvraag}

De centrale onderzoeksvraag van dit onderzoek luidt: "Hoe kunnen Edge Database-architecturen geoptimaliseerd worden om efficiënte gegevensopslag in Edge-omgevingen te faciliteren?" Deze vraag richt zich op het verkennen van nieuwe inzichten en oplossingen voor gegevensbeheer in Edge-omgevingen.

\section{\IfLanguageName{dutch}{Onderzoeksdoelstelling}{Research objective}}%
\label{sec:onderzoeksdoelstelling}

Het doel van dit onderzoek is niet alleen om praktische aanbevelingen te formuleren, maar ook om nieuwe inzichten en kennis op het gebied van gegevensopslag in Edge-omgevingen te verkennen. Het beoogde resultaat is een diepgaand begrip van Edge Database-architecturen en hun optimalisatiestrategieën.

\section{\IfLanguageName{dutch}{Opzet van deze bachelorproef}{Structure of this bachelor thesis}}%
\label{sec:opzet-bachelorproef}

De rest van deze bachelorproef is als volgt opgebouwd:

In Hoofdstuk 2 wordt een overzicht gegeven van de stand van zaken binnen het onderzoeksdomein, op basis van een literatuurstudie.

In Hoofdstuk 3 wordt de methodologie toegelicht en worden de gebruikte onderzoekstechnieken besproken om een antwoord te kunnen formuleren op de onderzoeksvragen.

In Hoofdstuk 4 wordt de conclusie gegeven en een antwoord geformuleerd op de onderzoeksvragen. Daarbij wordt ook een aanzet gegeven voor toekomstig onderzoek binnen dit domein.