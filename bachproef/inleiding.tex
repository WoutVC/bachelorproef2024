%%=============================================================================
%% Inleiding
%%=============================================================================

\chapter{\IfLanguageName{dutch}{Inleiding}{Introduction}}%
\label{ch:inleiding}

In deze inleiding geef ik een overzicht van het onderwerp van mijn bachelorproef, waarin ik de impact van data-partitioneringstechnieken op de prestaties van databases.
Het doel is om te evalueren hoe deze technieken kunnen worden ingezet in een eenvoudige en praktische Edge Computing-context die relevant is voor Optis, een consultancybedrijf gespecialiseerd in IT-oplossingen. Als concrete use case wordt het optimaliseren van data-opslag en verwerking in een IoT-omgeving onderzocht, waarbij uitdagingen zoals hoge latentie en inefficiënt bandbreedtegebruik aangepakt worden.

\section{\IfLanguageName{dutch}{Probleemstelling}{Problem Statement}}%
\label{sec:probleemstelling}

De kernprobleemstelling van deze studie is het verbeteren van de efficiëntie van gegevensopslag en -verwerking in een IoT-omgeving. 
Optis ondersteunt klanten bij het implementeren van IT-oplossingen in gedistribueerde systemen. Veel van deze klanten werken met IoT-apparaten zoals sensoren en monitoringtools. Deze apparaten genereren grote hoeveelheden gegevens die moeten worden opgeslagen en verwerkt. 
Dit leidt vaak tot problemen zoals lange responstijden, inefficiënte data-overdracht en schaalbaarheidsbeperkingen. Deze uitdagingen vereisen oplossingen die gericht zijn op lokaal databeheer, ondersteund door effectieve data-partitioneringstechnieken.

\section{\IfLanguageName{dutch}{Onderzoeksvraag}{Research question}}%
\label{sec:onderzoeksvraag}

De centrale onderzoeksvraag die in deze studie wordt behandeld, is: 
\\ "Hoe beïnvloeden data-partitioneringstechnieken in combinatie met edge-databases de prestaties \\ van een Edge Computing omgeving van Optis?" \\ 
  Deze vraag richt zich op hoe data-partitioneringstechnieken de prestaties van edge-databases beïnvloeden binnen een Edge Computing-omgeving. Het doel van het onderzoek is om zowel de prestaties van databases te verbeteren als een geschikte edge-database te identificeren voor gebruik in deze context.

\section{\IfLanguageName{dutch}{Onderzoeksdoelstelling}{Research objective}}%
\label{sec:onderzoeksdoelstelling}

Het doel van dit onderzoek is om praktische aanbevelingen te formuleren voor Optis door de impact van verschillende data-partitioneringstechnieken op de prestaties van databases te analyseren binnen een eenvoudige simulatie. Deze studie zal bijdragen aan het begrijpen van de toepasbaarheid van partitioneringstechnieken in een kleine, schaalbare context en zal praktische inzichten bieden die door Optis verder kunnen worden toegepast.

\begin{itemize}
  \item Het uitvoeren van een gedetailleerde evaluatie van de impact van data-partitioneringstechnieken op de prestaties van edge-databases in een IoT-omgeving.
  \item Het ontwikkelen van een proof of concept om te onderzoeken hoe deze technieken de latentie, bandbreedte en schaalbaarheid beïnvloeden in een Edge Computing-context.
  \item Het formuleren van praktische aanbevelingen voor Optis op basis van de resultaten van de proof of concept.
\end{itemize}

\section{\IfLanguageName{dutch}{Opzet van deze bachelorproef}{Structure of this bachelor thesis}}%
\label{sec:opzet-bachelorproef}

De rest van deze bachelorproef is als volgt opgebouwd:

In Hoofdstuk 2 wordt een overzicht gegeven van de stand van zaken binnen het onderzoeksdomein door middel van een literatuurstudie. Deze studie richt zich op de mogelijkheden en uitdagingen van Edge Computing, relevante data-partitioneringstechnieken, en de prestaties van bestaande Edge-databases.

In Hoofdstuk 3 wordt de methodologie besproken. Hier worden de gekozen onderzoeksmethoden en gebruikte technieken toegelicht, waaronder de proof of concept om de impact van data-partitioneringstechnieken te evalueren.

In Hoofdstuk 4 worden de resultaten gepresenteerd en geanalyseerd, waarbij een antwoord wordt geformuleerd op de onderzoeksvraag en aanbevelingen voor toekomstig onderzoek in dit domein worden besproken.

\section{\IfLanguageName{dutch}{Deelvragen}{Sub-questions}}%
\label{sec:deelvragen}

Om de onderzoeksvraag verder te verduidelijken en te structureren, worden de volgende deelvragen behandeld:
\begin{itemize}
    \item Hoe verhouden verschillende data-partitioneringstechnieken zich in termen van latentie in een Edge-omgeving?
    \item Wat is de invloed van data-partitioneringstechnieken op netwerkprestaties in gedistribueerde Edge-netwerken?
    \item Hoe beïnvloeden deze technieken de schaalbaarheid van databases in een dynamische Edge-omgeving?
    \item Welke bestaande Edge-databasetechnologieën leveren, gemeten aan de hand van latentie, fouttolerantie en schaalbaarheid, de beste prestaties in combinatie met de onderzochte partitioneringstechnieken binnen de context van Optis?
\end{itemize}
