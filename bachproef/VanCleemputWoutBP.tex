ç!%===============================================================================
% LaTeX sjabloon voor de bachelorproef toegepaste informatica aan HOGENT
% Meer info op https://github.com/HoGentTIN/latex-hogent-report
%===============================================================================

\documentclass[dutch,dit,thesis]{hogentreport}

\usepackage{tikz}
\usepackage{pgfgantt}

% TODO:
% - If necessary, replace the option `dit`' with your own department!
%   Valid entries are dbo, dbt, dgz, dit, dlo, dog, dsa, soa
% - If you write your thesis in English (remark: only possible after getting
%   explicit approval!), remove the option "dutch," or replace with "english".

\usepackage{lipsum} % For blind text, can be removed after adding actual content

%% Pictures to include in the text can be put in the graphics/ folder
\graphicspath{{graphics/}}

%% For source code highlighting, requires pygments to be installed
%% Compile with the -shell-escape flag!
\usepackage[section]{minted}
%% If you compile with the make_thesis.{bat,sh} script, use the following
%% import instead:
%% \usepackage[section,outputdir=../output]{minted}
\usemintedstyle{solarized-light}
\definecolor{bg}{RGB}{253,246,227} %% Set the background color of the codeframe

%% Change this line to edit the line numbering style:
\renewcommand{\theFancyVerbLine}{\ttfamily\scriptsize\arabic{FancyVerbLine}}

%% Macro definition to load external java source files with \javacode{filename}:
\newmintedfile[javacode]{java}{
    bgcolor=bg,
    fontfamily=tt,
    linenos=true,
    numberblanklines=true,
    numbersep=5pt,
    gobble=0,
    framesep=2mm,
    funcnamehighlighting=true,
    tabsize=4,
    obeytabs=false,
    breaklines=true,
    mathescape=false
    samepage=false,
    showspaces=false,
    showtabs =false,
    texcl=false,
}

% Other packages not already included can be imported here

%%---------- Document metadata -------------------------------------------------
% TODO: Replace this with your own information
\author{Wout Van Cleemput}
\supervisor{Dhr. F. Van Houte}
\cosupervisor{Mevr. S. Beeckman}
\title[Optionele ondertitel]%
    {Optimalisatie van gegevensopslag in Edge Database-architecturen: Een diepgaande verkenning \\ van efficiënte oplossingen in Edge-omgevingen}
\academicyear{\advance\year by -1 \the\year--\advance\year by 1 \the\year}
\examperiod{1}
\degreesought{\IfLanguageName{dutch}{Professionele bachelor in de toegepaste informatica}{Bachelor of applied computer science}}
\partialthesis{false} %% To display 'in partial fulfilment'
%\institution{Internshipcompany BVBA.}

%% Add global exceptions to the hyphenation here
\hyphenation{back-slash}

%% The bibliography (style and settings are  found in hogentthesis.cls)
\addbibresource{bachproef.bib}            %% Bibliography file
\addbibresource{../voorstel/voorstel.bib} %% Bibliography research proposal
\defbibheading{bibempty}{}

%% Prevent empty pages for right-handed chapter starts in twoside mode
\renewcommand{\cleardoublepage}{\clearpage}

\renewcommand{\arraystretch}{1.2}

%% Content starts here.
\begin{document}

%---------- Front matter -------------------------------------------------------

\frontmatter

\hypersetup{pageanchor=false} %% Disable page numbering references
%% Render a Dutch outer title page if the main language is English
\IfLanguageName{english}{%
    %% If necessary, information can be changed here
    \degreesought{Professionele Bachelor toegepaste informatica}%
    \begin{otherlanguage}{dutch}%
       \maketitle%
    \end{otherlanguage}%
}{}

%% Generates title page content
\maketitle
\hypersetup{pageanchor=true}

%%=============================================================================
%% Voorwoord
%%=============================================================================

\chapter*{\IfLanguageName{dutch}{Woord vooraf}{Preface}}%
\label{ch:voorwoord}

%% TODO:
%% Het voorwoord is het enige deel van de bachelorproef waar je vanuit je
%% eigen standpunt (``ik-vorm'') mag schrijven. Je kan hier bv. motiveren
%% waarom jij het onderwerp wil bespreken.
%% Vergeet ook niet te bedanken wie je geholpen/gesteund/... heeft

Met veel enthousiasme presenteer ik mijn bachelorproef over de optimalisatie van gegevensopslag in Edge-omgevingen.

Dit onderwerp heeft mijn aandacht getrokken vanwege de opkomst van Edge computing en omdat ik mijn kennis over databases wou uitbreiden.

Het verkennen van Edge Database-architecturen biedt een fascinerende kijk op de evolutie van gegevensbeheer, en de impact die deze nieuwe technologieën hebben op de wereld van IT.

Ik wil graag mijn co-promotor, Kenneth Van den Berghe van het bedrijf optis en mijn begeleider Lieven Smits, bedanken voor hun begeleiding en feedback tijdens het schrijven van deze bachelorproef.

Ook wil ik mijn familie en vrienden bedanken voor hun steun en aanmoediging tijdens het schrijfproces.

Ik hoop dat deze bachelorproef een waardevolle bijdrage zal leveren aan het onderzoeksdomein en dat het lezers zal overtuigen om verder te gaan met dit interessante onderwerp.

%%=============================================================================
%% Samenvatting
%%=============================================================================

% TODO: De "abstract" of samenvatting is een kernachtige (~ 1 blz. voor een
% thesis) synthese van het document.
%
% Een goede abstract biedt een kernachtig antwoord op volgende vragen:
%
% 1. Waarover gaat de bachelorproef?
% 2. Waarom heb je er over geschreven?
% 3. Hoe heb je het onderzoek uitgevoerd?
% 4. Wat waren de resultaten? Wat blijkt uit je onderzoek?
% 5. Wat betekenen je resultaten? Wat is de relevantie voor het werkveld?
%
% Daarom bestaat een abstract uit volgende componenten:
%
% - inleiding + kaderen thema
% - probleemstelling
% - (centrale) onderzoeksvraag
% - onderzoeksdoelstelling
% - methodologie
% - resultaten (beperk tot de belangrijkste, relevant voor de onderzoeksvraag)
% - conclusies, aanbevelingen, beperkingen
%
% LET OP! Een samenvatting is GEEN voorwoord!

%%---------- Nederlandse samenvatting -----------------------------------------
%
% TODO: Als je je bachelorproef in het Engels schrijft, moet je eerst een
% Nederlandse samenvatting invoegen. Haal daarvoor onderstaande code uit
% commentaar.
% Wie zijn bachelorproef in het Nederlands schrijft, kan dit negeren, de inhoud
% wordt niet in het document ingevoegd.

\IfLanguageName{english}{%
\selectlanguage{dutch}
\chapter*{Samenvatting}
\lipsum[1-4]
\selectlanguage{english}
}{}

%%---------- Samenvatting -----------------------------------------------------
% De samenvatting in de hoofdtaal van het document

\chapter*{\IfLanguageName{dutch}{Samenvatting}{Abstract}}

\lipsum[1-4]


%---------- Inhoud, lijst figuren, ... -----------------------------------------

\tableofcontents

% In a list of figures, the complete caption will be included. To prevent this,
% ALWAYS add a short description in the caption!
%
%  \caption[short description]{elaborate description}
%
% If you do, only the short description will be used in the list of figures

\listoffigures

% If you included tables and/or source code listings, uncomment the appropriate
% lines.
%\listoftables
%\listoflistings

% Als je een lijst van afkortingen of termen wil toevoegen, dan hoort die
% hier thuis. Gebruik bijvoorbeeld de ``glossaries'' package.
% https://www.overleaf.com/learn/latex/Glossaries

%---------- Kern ---------------------------------------------------------------

\mainmatter{}

% De eerste hoofdstukken van een bachelorproef zijn meestal een inleiding op
% het onderwerp, literatuurstudie en verantwoording methodologie.
% Aarzel niet om een meer beschrijvende titel aan deze hoofdstukken te geven of
% om bijvoorbeeld de inleiding en/of stand van zaken over meerdere hoofdstukken
% te verspreiden!

%%=============================================================================
%% Inleiding
%%=============================================================================

\chapter{\IfLanguageName{dutch}{Inleiding}{Introduction}}%
\label{ch:inleiding}

In deze inleiding geef ik een overzicht van het onderwerp van mijn bachelorproef, waarin ik de impact van data-partitioneringstechnieken op de prestaties van databases.
Het doel is om te evalueren hoe deze technieken kunnen worden ingezet in een eenvoudige en praktische Edge Computing-context die relevant is voor Optis, een consultancybedrijf gespecialiseerd in IT-oplossingen. Als concrete use case wordt het optimaliseren van data-opslag en verwerking in een IoT-omgeving onderzocht, waarbij uitdagingen zoals hoge latentie en inefficiënt bandbreedtegebruik aangepakt worden.

\section{\IfLanguageName{dutch}{Probleemstelling}{Problem Statement}}%
\label{sec:probleemstelling}

De kernprobleemstelling van deze studie is het verbeteren van de efficiëntie van gegevensopslag en -verwerking in een IoT-omgeving. 
Optis ondersteunt klanten bij het implementeren van IT-oplossingen in gedistribueerde systemen. Veel van deze klanten werken met IoT-apparaten zoals sensoren en monitoringtools. Deze apparaten genereren grote hoeveelheden gegevens die moeten worden opgeslagen en verwerkt. 
Dit leidt vaak tot problemen zoals lange responstijden, inefficiënte data-overdracht en schaalbaarheidsbeperkingen. Deze uitdagingen vereisen oplossingen die gericht zijn op lokaal databeheer, ondersteund door effectieve data-partitioneringstechnieken.

\section{\IfLanguageName{dutch}{Onderzoeksvraag}{Research question}}%
\label{sec:onderzoeksvraag}

De centrale onderzoeksvraag die in deze studie wordt behandeld, is: 
\\ "Hoe beïnvloeden data-partitioneringstechnieken in combinatie met edge-databases de prestaties \\ van een Edge Computing omgeving van Optis?" \\ 
  Deze vraag richt zich op hoe data-partitioneringstechnieken de prestaties van edge-databases beïnvloeden binnen een Edge Computing-omgeving. Het doel van het onderzoek is om zowel de prestaties van databases te verbeteren als een geschikte edge-database te identificeren voor gebruik in deze context.

\section{\IfLanguageName{dutch}{Onderzoeksdoelstelling}{Research objective}}%
\label{sec:onderzoeksdoelstelling}

Het doel van dit onderzoek is om praktische aanbevelingen te formuleren voor Optis door de impact van verschillende data-partitioneringstechnieken op de prestaties van databases te analyseren binnen een eenvoudige simulatie. Deze studie zal bijdragen aan het begrijpen van de toepasbaarheid van partitioneringstechnieken in een kleine, schaalbare context en zal praktische inzichten bieden die door Optis verder kunnen worden toegepast.

\begin{itemize}
  \item Het uitvoeren van een gedetailleerde evaluatie van de impact van data-partitioneringstechnieken op de prestaties van edge-databases in een IoT-omgeving.
  \item Het ontwikkelen van een proof of concept om te onderzoeken hoe deze technieken de latentie, bandbreedte en schaalbaarheid beïnvloeden in een Edge Computing-context.
  \item Het formuleren van praktische aanbevelingen voor Optis op basis van de resultaten van de proof of concept.
\end{itemize}

\section{\IfLanguageName{dutch}{Opzet van deze bachelorproef}{Structure of this bachelor thesis}}%
\label{sec:opzet-bachelorproef}

De rest van deze bachelorproef is als volgt opgebouwd:

In Hoofdstuk 2 wordt een overzicht gegeven van de stand van zaken binnen het onderzoeksdomein door middel van een literatuurstudie. Deze studie richt zich op de mogelijkheden en uitdagingen van Edge Computing, relevante data-partitioneringstechnieken, en de prestaties van bestaande Edge-databases.

In Hoofdstuk 3 wordt de methodologie besproken. Hier worden de gekozen onderzoeksmethoden en gebruikte technieken toegelicht, waaronder de proof of concept om de impact van data-partitioneringstechnieken te evalueren.

In Hoofdstuk 4 worden de resultaten gepresenteerd en geanalyseerd, waarbij een antwoord wordt geformuleerd op de onderzoeksvraag en aanbevelingen voor toekomstig onderzoek in dit domein worden besproken.

\section{\IfLanguageName{dutch}{Deelvragen}{Sub-questions}}%
\label{sec:deelvragen}

Om de onderzoeksvraag verder te verduidelijken en te structureren, worden de volgende deelvragen behandeld:
\begin{itemize}
    \item Hoe verhouden verschillende data-partitioneringstechnieken zich in termen van latentie in een Edge-omgeving?
    \item Wat is de invloed van data-partitioneringstechnieken op bandbreedteverbruik en netwerkprestaties in gedistribueerde Edge-netwerken?
    \item Hoe beïnvloeden deze technieken de schaalbaarheid van databases in een dynamische Edge-omgeving?
    \item Welke bestaande Edge-databasetechnologieën leveren, gemeten aan de hand van latentie, fouttolerantie en schaalbaarheid, de beste prestaties in combinatie met de onderzochte partitioneringstechnieken binnen de context van Optis?
\end{itemize}

\chapter{\IfLanguageName{dutch}{Stand van zaken}{State of the art}}%
\label{ch:stand-van-zaken}

Dit hoofdstuk biedt een uitgebreide literatuurstudie over EdgeDB als een tijdreeksdatabase voor het beheren van grote hoeveelheden tijdreeksgegevens, als onderdeel van deze bachelorproef.

Het is bedoeld om een grondige analyse te geven van de huidige stand van zaken binnen het onderzoeksdomein.

De inhoud van dit hoofdstuk bouwt voort op de inleiding en heeft als doel de lezer volledig op de hoogte te brengen van recente ontwikkelingen, technologieën en benaderingen die relevant zijn voor dit onderwerp, zodat zij het verdere verloop van de bachelorproef kunnen begrijpen zonder verdere opzoekingen te moeten doen.

\section{Bronnen zoekmethode}
De bronnen voor deze literatuurstudie werden gevonden door middel van een systematische zoekmethode, waarbij gebruik werd gemaakt van verschillende zoekmachines en databases, waaronder Google Scholar, IEEE Xplore, Semanthic Scolar en Elicit.
Als zoektermen gebruikten we "EdgeDB", "Database on Edge" en "Edge Computing".

\section{Database on Edge}

Database on Edge is een opkomende benadering binnen database technologie, voortkomend uit het gebruik van Edge computing \autocite{Yang2019EdgeDBAE}.

In tegenstelling tot traditionele databases worden Edge databases verspreid over verschillende lokale apparaten, waardoor gegevens lokaal verwerkt en opgeslagen kunnen worden, in plaats van centraal in de cloud.

Deze innovatieve benadering richt zich, vergelijkbaar met moderne bedrijven die streven naar efficiëntie en innovatie, op het verminderen van latency en het verbeteren van de prestaties.

Het concept is ontstaan uit de noodzaak om real-time interacties mogelijk te maken, wat cruciaal is in diverse toepassingen \autocite{Yang2019EdgeDBAE}.

Voordelen van Database on Edge omvatten lokale gegevensverwerking, verminderde latency en verbeterde prestaties, waardoor het geschikt is voor toepassingen waar snelle respons cruciaal is.

Echter, zoals bij elke technologische benadering, zijn er uitdagingen verbonden aan Database on Edge, waaronder het beheer van diverse databases over verschillende apparaten, optimalisatie van resourcegebruik op Edge Devices, en het waarborgen van consistente prestaties.

De optimalisatie van Database on Edge is een actief onderzoeksgebied en omvat het verkennen van efficiënte algoritmen voor gegevensverwerking, verbeteringen in gegevensopslag, en het beheer van gedistribueerde databases om de prestaties te maximaliseren.

\section{Tijdreeksdatabases}

Tijdreeksdatabases zijn van essentieel belang in moderne informatiesystemen, met toepassingen variërend van IoT (Internet of Things) tot financiële analyses en meer.

Dit gespecialiseerde type database is geoptimaliseerd voor het opslaan, beheren en analyseren van tijdreeksgegevens, waarbij elk gegevenspunt wordt geassocieerd met een tijdstempel. 

In de afgelopen jaren zijn verschillende tijdreeksdatabases ontwikkeld om te voldoen aan de specifieke behoeften van verschillende toepassingsgebieden.

Voorbeelden hiervan zijn BTrDB en InfluxDB, twee bekende tijdreeksdatabases die elk hun eigen benaderingen en optimalisaties bieden voor het verwerken van tijdreeksgegevens \autocite{Yang2019EdgeDBAE}.

\section{EdgeDB: Een nieuwe benadering voor tijdreeksdatabases}

EdgeDB is een recente toevoeging aan het landschap van tijdreeksdatabases en belooft een innovatieve benadering te bieden voor het beheren van tijdreeksgegevens.

Deze nieuwe database introduceert nieuwe indexstructuren en queryverwerkingstechnieken om de prestaties en efficiëntie te verbeteren.

Een van de belangrijkste kenmerken van EdgeDB is het gebruik van TMTree, een geoptimaliseerde indexstructuur die is ontworpen om schrijfoperaties te optimaliseren en het geheugengebruik te minimaliseren \autocite{Yang2019EdgeDBAE}.

\section{Vergelijkende evaluatie van EdgeDB}

Een grondige evaluatie van EdgeDB is essentieel om de prestaties en bruikbaarheid van deze nieuwe tijdreeksdatabase te begrijpen in vergelijking met bestaande oplossingen zoals BTrDB en InfluxDB.

Verschillende evaluatiemetrics worden gebruikt, waaronder insert-prestaties, schrijfprestaties, query-prestaties, geheugenoverhead en invoersnelheid van TMTree.

Deze evaluatie biedt inzicht in de sterke punten en beperkingen van EdgeDB en helpt bij het bepalen van de geschiktheid ervan voor verschillende toepassingsgebieden en gebruiksscenario's \autocite{Yang2019EdgeDBAE}.

\section{VergeDB: Een innovatieve tijdreeksdatabase voor Edge Computing}

Een recente ontwikkeling op het gebied van tijdreeksdatabases is VergeDB, een database die specifiek is ontworpen voor IoT-analytics op Edge-apparaten \autocite{Paparrizos2021VergeDBAD}. VergeDB belooft een innovatieve benadering te bieden voor het beheren van tijdreeksgegevens op de rand van het netwerk, wat cruciaal is voor real-time toepassingen waarbij snelle respons vereist is.

Deze nieuwe database introduceert nieuwe compressiemethoden, indexstructuren en queryverwerkingstechnieken om de prestaties en efficiëntie te verbeteren. Een van de belangrijkste kenmerken van VergeDB is de mogelijkheid om gegevens lokaal op te slaan en te verwerken op Edge-apparaten, waardoor de latentie wordt verminderd en de prestaties worden verbeterd.

VergeDB biedt ook ondersteuning voor geavanceerde analysetaken en machine learning-taken, zoals trendanalyse, patroonherkenning en anomaliedetectie, waardoor het een veelzijdige oplossing is voor diverse toepassingen op het gebied van IoT-analytics.

Deze nieuwe benadering van tijdreeksdatabases op de rand van het netwerk opent nieuwe mogelijkheden voor het efficiënt beheren en analyseren van tijdreeksgegevens in real-time, wat essentieel is voor het succes van IoT-toepassingen in verschillende domeinen.

\section{Conclusie}
Database on Edge, voortkomend uit het gebruik van Edge computing, biedt veelbelovende voordelen zoals lokale gegevensverwerking, verminderde latency en verbeterde prestaties. Echter, er zijn ook uitdagingen verbonden aan deze benadering, zoals het beheer van diverse databases over verschillende apparaten en het optimaliseren van resourcegebruik op Edge Devices.

EdgeDB en VergeDB vertegenwoordigen nieuwe benaderingen voor het beheren van tijdreeksgegevens op de rand van het netwerk, met innovatieve functies zoals geoptimaliseerde indexstructuren en queryverwerkingstechnieken. Deze benaderingen openen nieuwe mogelijkheden voor het efficiënt beheren en analyseren van tijdreeksgegevens in real-time, wat essentieel is voor het succes van IoT-toepassingen in verschillende domeinen.

De volgende stap in dit onderzoek is de methodologie voor de vergelijkende evaluatie van EdgeDB te beschrijveb. Dit zal helpen bij het beoordelen van de prestaties en bruikbaarheid van EdgeDB in vergelijking met andere tijdreeksdatabases, en zal verdere inzichten bieden in de geschiktheid ervan voor verschillende toepassingsgebieden en gebruiksscenario's.

%%=============================================================================
%% Methodologie
%%=============================================================================

\chapter{\IfLanguageName{dutch}{Methodologie}{Methodology}}%
\label{ch:methodologie}

%% TODO: In dit hoofstuk geef je een korte toelichting over hoe je te werk bent
%% gegaan. Verdeel je onderzoek in grote fasen, en licht in elke fase toe wat
%% de doelstelling was, welke deliverables daar uit gekomen zijn, en welke
%% onderzoeksmethoden je daarbij toegepast hebt. Verantwoord waarom je
%% op deze manier te werk gegaan bent.
%% 
%% Voorbeelden van zulke fasen zijn: literatuurstudie, opstellen van een
%% requirements-analyse, opstellen long-list (bij vergelijkende studie),
%% selectie van geschikte tools (bij vergelijkende studie, "short-list"),
%% opzetten testopstelling/PoC, uitvoeren testen en verzamelen
%% van resultaten, analyse van resultaten, ...
%%
%% !!!!! LET OP !!!!!
%%
%% Het is uitdrukkelijk NIET de bedoeling dat je het grootste deel van de corpus
%% van je bachelorproef in dit hoofstuk verwerkt! Dit hoofdstuk is eerder een
%% kort overzicht van je plan van aanpak.
%%
%% Maak voor elke fase (behalve het literatuuronderzoek) een NIEUW HOOFDSTUK aan
%% en geef het een gepaste titel.

In dit hoofdstuk wordt de aanpak van het onderzoek toegelicht. 
 Om de onderzoeksvraag te beantwoorden wordt een combinatie van literatuurstudie, een vergelijkende analyse en een proof of concept toegepast.

De proof of concept richt zich op het implementeren en testen van databases die gebruikmaken van verschillende partitioneringstechnieken.
 De prestaties van deze databases worden gemeten op basis van gedefinieerde criteria zoals snelheid, schaalbaarheid en betrouwbaarheid in een Edge-omgeving.

De technische diepte van deze bachelorproef wordt benadrukt,
 inclusief de beschrijving van gebruikte tools zoals hardware, software en diensten. \\

Een gedetailleerde tijdsplanning wordt opgesteld om de duur van elke onderzoeksfase en de concrete deliverables te definiëren.
Deze tijdsplanning wordt opgesteld in de vorm van een Gantt chart en een flowchart. Deze zal u vinden op de laatste pagina van dit voorstel.

Het onderzoek verloopt in vier fases, die elk gericht zijn op bepaalde doelstellingen en methoden. \\

\section*{Fase 1: Literatuurstudie}

\textbf{Doelstelling:}  
Inzicht krijgen in data-partitioneringstechnieken, vereisten van edge-databases en toepassingen in sensornetwerken.

\textbf{Aanpak:}
\begin{itemize}
    \item Analyse van wetenschappelijke literatuur en officiële documentatie.
    \item Identificatie van functionele en niet-functionele vereisten voor databases in edge-omgevingen.
    \item Opstellen van een longlist van mogelijke databases en technieken.
\end{itemize}

\textbf{Beantwoorde deelvragen:}
\begin{itemize}
    \item \textbf{Welke uitdagingen ontstaan bij het verwerken van grote hoeveelheden IoT-data in een centrale cloudomgeving?} \\ 
    Literatuur toont aan waar cloudverwerking tekortschiet in netwerken met veel sensoren.
    \item \textbf{Wat zijn de functionele vereisten van databases voor het verwerken van real-time sensordata in een edge-context met beperkte connectiviteit?} \\
    Deze vraag wordt beantwoord aan de hand van een literatuurstudie naar bestaande databaseoplossingen die geschikt zijn voor gebruik in edge-omgevingen met sensornetwerken.
\end{itemize}

\textbf{Deliverable:} Een overzicht van de belangrijkste inzichten omtrent partitionering, edge computing en real-time monitoring. \\
\textbf{Deadline:} 21/11/2024

\section*{Fase 2: Selectie en vergelijking van databases}

\textbf{Doelstelling:}  
Databasetechnologieën identificeren die geschikt zijn voor edge-omgevingen en deze theoretisch vergelijken op basis van relevante criteria.

\textbf{Aanpak:}
\begin{itemize}
    \item Opstellen van evaluatiecriteria zoals latentie, fouttolerantie, schaalbaarheid en gebruiksgemak.
    \item Selectie van drie geschikte databases: Cassandra, MongoDB en TimescaleDB.
    \item Vergelijkende analyse van deze databases aan de hand van documentatie en bestaande benchmarks.
\end{itemize}

\textbf{Beantwoorde deelvraag:}
\begin{itemize}
    \item \textbf{Welke van de geselecteerde edge-databases leveren de beste prestaties op vlak van latentie, fouttolerantie en schaalbaarheid binnen de HoGent-use case?} \\
    De vergelijkende studie onderzoekt de prestaties van de databases en hun mogelijke geschiktheid voor de use case.
\end{itemize}

\textbf{Deliverable:} Een gemotiveerde shortlist met drie geschikte edge-databases. \\
\textbf{Deadline:} 05/12/2024

\section*{Fase 3: Proof of Concept}

\textbf{Doelstelling:}  
De geselecteerde databases implementeren in een gesimuleerde edge-omgeving en hun prestaties testen met voorbeeldgegevens.

\textbf{Aanpak:}
\begin{itemize}
    \item Opzetten van een testomgeving met Docker Compose die een edge-scenario nabootst.
    \item Genereren van testdata die sensorwaardes simuleert (CO₂, druk, temperatuur om de 5 seconden).
    \item Implementeren van list-based en range-based partitionering.
    \item Testen op latency, schaalbaarheid, foutafhandeling en beschikbaarheid.
    \item Verzamelen en analyseren van meetgegevens via scripts.
\end{itemize}

\textbf{Beantwoorde deelvragen:}
\begin{itemize}
    \item \textbf{Hoe beïnvloeden verschillende data-partitioneringstechnieken de latentie bij dataverwerking in een Edge Computing-omgeving?} \\
    Door latency te meten in verschillende opstellingen met verschillende technieken.
    \item \textbf{Wat is het effect van verschillende data-partitioneringstechnieken op de netwerkbelasting bij het verwerken van real-time sensordata in een gedistribueerd sensornetwerk?} \\
    Netwerkverkeer wordt gemeten tijdens het testen van de technieken.
    \item \textbf{Wat is de impact van partitioneringsstrategieën op de schaalbaarheid van edge-databases bij toenemende datavolumes?} \\
    Door het effect van groeiende datasets te analyseren bij elke techniek.
\end{itemize}

\textbf{Deliverable:} Een rapport met testresultaten, grafieken en evaluatie per techniek en database. \\
\textbf{Deadline:} 24/12/2024

\section*{Fase 4: Analyse en evaluatie}

\textbf{Doelstelling:}  
Conclusies trekken over welke database en partitioneringstechniek de beste prestaties leveren in deze edge-context.

\textbf{Aanpak:}
\begin{itemize}
    \item Analyse van testresultaten uit de PoC.
    \item Vergelijking op basis van prestatiecriteria.
    \item Formuleren van aanbevelingen voor toekomstige toepassingen.
    \item Bepalen van een totaalscore per database aan de hand van genormaliseerde prestatie-indicatoren, waarbij elke indicator gewogen wordt op basis van de relevantie voor edge computing.
\end{itemize}

\textbf{Beantwoorde deelvraag:}
\begin{itemize}
    \item \textbf{Welke van de geselecteerde edge-databases leveren de beste prestaties op vlak van latentie, fouttolerantie en schaalbaarheid binnen de HoGent-use case?} \\
    De analyse van de testresultaten toont welke oplossing het best past bij de gesimuleerde toepassing.
\end{itemize}

\textbf{Deliverable:} Een eindrapport met visuele samenvattingen, conclusies en aanbevelingen. Inclusief een rangschikking van alle databases op basis van genormaliseerde prestatiecriteria en bijbehorende wegingsfactoren. \\
\textbf{Deadline:} 05/01/2025

% Voeg hier je eigen hoofdstukken toe die de ``corpus'' van je bachelorproef
% vormen. De structuur en titels hangen af van je eigen onderzoek. Je kan bv.
% elke fase in je onderzoek in een apart hoofdstuk bespreken.

%\input{...}
%\input{...}
%...

Om de onderzoeksvraag te beantwoorden, gaan we gebruik maken van een combinatie van literatuurstudie,
 interviews met relevante bedrijven en een grondige vergelijkende studie. \\
 
Specifieke methoden, zoals requirements-analyse en experimenten, worden toegepast.

We gaan geen enquetes uitvoeren, omdat we de voorkeur geven aan een meer persoonlijke aanpak. \\
 
De technische diepte van deze bachelorproef wordt benadrukt,
 inclusief de beschrijving van gebruikte tools zoals hardware, software en \\ diensten. \\

Een gedetailleerde tijdsplanning wordt \\ opgesteld om de duur van elke onderzoeksfase en de concrete deliverables te definiëren.
Deze tijdsplanning wordt opgesteld in de vorm van een Gantt chart. Deze zal u vinden op de laatste \\ pagina van dit voorstel.


%%=============================================================================
%% Conclusie
%%=============================================================================

\chapter{Conclusie}%
\label{ch:conclusie}

% TODO: Trek een duidelijke conclusie, in de vorm van een antwoord op de
% onderzoeksvra(a)g(en). Wat was jouw bijdrage aan het onderzoeksdomein en
% hoe biedt dit meerwaarde aan het vakgebied/doelgroep? 
% Reflecteer kritisch over het resultaat. In Engelse teksten wordt deze sectie
% ``Discussion'' genoemd. Had je deze uitkomst verwacht? Zijn er zaken die nog
% niet duidelijk zijn?
% Heeft het onderzoek geleid tot nieuwe vragen die uitnodigen tot verder 
%onderzoek?

In dit onderzoek werd nagegaan hoe data-partitioneringstechnieken en edge-databases de prestaties van real-time dataverwerking beïnvloeden binnen een Edge Computing-architectuur. De proof of concept werd uitgevoerd in een gesimuleerd scenario gebaseerd op een HoGent-use case, waarin elk lokaal via een edge-node sensordata verzamelt over \ce{CO2}, temperatuur en luchtdruk.

De resultaten tonen aan dat een edge-gebaseerde aanpak aanzienlijke voordelen biedt ten opzichte van centrale verwerking. Door data lokaal te verwerken en op te slaan, daalt de latentie en wordt het netwerk minder belast. Partitioneringstechnieken bleken bovendien een bepalende factor te zijn voor databaseprestaties op het vlak van snelheid, schaalbaarheid en fouttolerantie.

\section{Vergelijking van databases}

De evaluatie van TimescaleDB, Cassandra en MongoDB in zowel gecentraliseerde als edge-opstellingen leverde de volgende inzichten op:

\begin{itemize}
    \item \textbf{TimescaleDB met range-based partitionering} behaalde de hoogste totaalscore (7.76). Deze configuratie scoorde goed op schaalbaarheid (5.67) en consistentie, en is zeer geschikt voor toepassingen met tijdreeksdata.
    \item \textbf{Cassandra met list-based partitionering} volgde kort daarna (7.74) en behaalde de laagste gemeten latentie (2.82 ms). Dit maakt Cassandra een sterke keuze in omgevingen waar reactiesnelheid en fouttolerantie belangrijk zijn.
    \item \textbf{MongoDB} had goede resultaten voor consistentie en throughput, maar scoorde het laagst op schaalbaarheid (1.95) en eindigde buiten de top drie.
\end{itemize}

\section{Beantwoording van de deelvragen}

\textbf{Welke uitdagingen ontstaan bij het verwerken van grote hoeveelheden IoT-data in een centrale cloudomgeving?} \\
Centrale verwerking zorgt voor hogere latentie, beperkte schaalbaarheid en is kwetsbaar bij netwerkproblemen. Edge Computing kan deze problemen deels oplossen.

\textbf{Wat zijn de functionele vereisten van databases voor het verwerken van real-time sensordata in een edge-context met beperkte connectiviteit?} \\
Databases moeten fouttolerant zijn, snel reageren (lage latentie), goed kunnen schalen, offline blijven werken en eenvoudig te beheren zijn in kleine, lokale systemen.

\textbf{Hoe beïnvloeden verschillende data-partitioneringstechnieken de latentie in een Edge Computing-omgeving?} \\
Cassandra met list-based partitionering had de laagste latentie (2.82 ms), gevolgd door MongoDB met list-based sharding (3.57 ms). De gekozen partitioneringstechniek heeft dus een duidelijke invloed op de reactiesnelheid.

\textbf{Wat is het effect van partitionering op netwerkbelasting en doorvoersnelheid?} \\
TimescaleDB met list-partitionering behaalde de hoogste throughput (4.86 records/s), gevolgd door andere TimescaleDB en MongoDB-configuraties. Zowel list als range-partitionering dragen bij aan een vlotte verwerking van data.

\textbf{Wat is de impact van partitioneringsstrategieën op schaalbaarheid bij toenemende datavolumes?} \\
TimescaleDB met range-partitionering scoorde het hoogst op schaalbaarheid (5.67), gevolgd door Cassandra met list-based partitionering (4.79). Deze technieken zijn dus geschikt voor grotere hoeveelheden data.

\textbf{Welke database biedt de beste prestaties in termen van latentie, fouttolerantie en schaalbaarheid in de HoGent-use case?} \\
De combinatie van goede schaalbaarheid, fouttolerantie en stabiele prestaties maakt TimescaleDB met range-partitionering de beste keuze. Cassandra is een goed alternatief wanneer vooral lage latentie belangrijk is.

\section{Reflectie en Aanbevelingen}

Deze studie bevestigt dat Edge Computing en data-partitionering real-time dataverwerking efficiënter en betrouwbaarder kunnen maken in IoT-toepassingen. De resultaten tonen aan dat databasekeuze en partitioneringstechniek samen gekozen moeten worden, afhankelijk van wat het belangrijkste is: snelheid, schaalbaarheid of fouttolerantie.

\textbf{Aanbevelingen voor verder onderzoek:}
\begin{itemize}
    \item Vergelijking met andere databaseoplossingen zoals Redis, InfluxDB of Apache IoTDB, inclusief hun geschiktheid voor gebruik in edge-omgevingen.
    \item Verkenning van hybride partitioneringsmodellen die automatisch kunnen schakelen op basis van datavolume of gebruik.
    \item Integratie van een Fog Computing-laag tussen edge en cloud, om data eerst lokaal te verwerken of samen te voegen voordat het naar de cloud gaat.
\end{itemize}

%---------- Bijlagen -----------------------------------------------------------

\appendix

\chapter{Onderzoeksvoorstel}

Het onderwerp van deze bachelorproef is gebaseerd op een onderzoeksvoorstel dat vooraf werd beoordeeld door de promotor. Dat voorstel is opgenomen in deze bijlage.

%% TODO: 
%\section*{Samenvatting}

% Kopieer en plak hier de samenvatting (abstract) van je onderzoeksvoorstel.
In deze bachelorproef gaan we een diepgaand onderzoek voeren naar efficiënte oplossingen voor het opslaan 
 van gegevens in Edge-omgevingen.
  We zullen dit doen door het concept van 'Edge Database' te verkennen
  en welke impact deze heeft op de wereld van IT.

Eerst moeten we grondig begrijpen wat de term 'Edge Database' inhoudt.

Van daaruit zullen we afleiden waarom dit concept de laatste tijd steeds meer aandacht krijgt
  en waarom zowel klanten als andere organisaties steeds meer interesse tonen in databases die specifiek ontworpen zijn voor
    Edge computing-omgevingen.
 
We zullen verder onderzoeken hoe bedrijven kunnen inspelen op de groeiende vraag naar efficiënte \\ gegevensopslag in Edge-omgevingen
  en welke softwareoplossingen beschikbaar zijn om deze databases te \\ optimaliseren.

Door deze aspecten grondig te bestuderen en samen te vatten in een literatuurstudie,
 beogen we deze \\ informatie te vergelijken met de praktijk door interviews af te nemen bij enkele relevante bedrijven.

We verwachten inzicht te krijgen in de behoefte aan Edge Database-oplossingen en willen specifiek onderzoeken welke rol
 verschillende technologieën spelen in het optimaliseren van gegevensopslag in Edge-omgevingen.

 Dit onderzoek zal niet enkel gebaseerd zijn op literatuurstudie, maar we gaan ook de samengevatte notities
  en antwoorden van de interviews met relevante bedrijven verwerken.

% Verwijzing naar het bestand met de inhoud van het onderzoeksvoorstel
%%---------- Inleiding ---------------------------------------------------------

\section{Introductie}%
\label{sec:introductie}

Deze bachelorproef richt zich op het onderzoek en begrijpen van de complexiteit rond gegevensopslag in Edge-omgevingen,
 met een specifieke focus op Edge Database-architecturen. \\
 
Het doel van dit onderzoek is om niet alleen praktische aanbevelingen te formuleren maar eerder het verkennen van
nieuwe inzichten en kennis op het gebied van gegevensopslag in \\ Edge-omgevingen. \\

We gaan niet enkel de bestaande informatie over dit onderwerp verzamelen, maar ook diepgaand onderzoek doen
 om nieuwe inzichten en oplossingen te bekomen. \\

De beoogde doelgroep van deze studie zijn werknemers die regelmatig werken met gegevensbeheer in Edge-omgevingen. \\

Het onderzoek is nauwkeurig afgestemd op een specifieke probleemsituatie binnen het vakgebied,
 gericht op het begrijpen en verbeteren van Edge Database-architecturen. \\

De kernprobleemstelling concentreert zich op het optimaliseren van gegevensopslag in Edge-omgevingen,
 met als centrale onderzoeksvraag: 
 "Hoe kunnen Edge Database-architecturen geoptimaliseerd worden om efficiënte gegevensopslag in Edge-omgevingen te faciliteren?" \\

Dit onderzoek bestudeert niet enkel de behoefte van Edge Database-architecturen,
maar ook de rol van verschillende technologieën
  in het optimaliseren van gegevensopslag in Edge-omgevingen. \\ \\

Dit vereist een uitgebreide analyse van de huidige stand van zaken en het verkennen van innovatieve oplossingen.

%---------- State of the art ---------------------------------------------------

\section{Stand van zaken}%
\label{sec:state-of-the-art}

Deze sectie onderzoekt de huidige stand van zaken met betrekking tot Database on Edge. \\

Hierbij ligt de focus op het begrijpen van de essentie, verschillen met traditionele databases, voordelen, uitdagingen,
 en optimalisatiemogelijkheden.

%---------- Database on Edge ---------------------------------------------------
\subsection{Database on Edge}%
\label{subsec:database_on_edge}

Database on Edge is een opkomende benadering binnen database technologie,
 voortkomend uit het gebruik van Edge computing  \cite{Yang2019EdgeDBAE}.  \\
 
In tegenstelling tot traditionele databases worden Edge databases verspreid over verschillende lokale apparaten,
 waardoor gegevens lokaal verwerkt en opgeslagen kunnen worden, in plaats van centraal in de cloud. \\

Deze innovatieve benadering richt zich, vergelijkbaar met moderne bedrijven die streven naar efficiëntie en innovatie,
 op het verminderen van latency en het verbeteren van de prestaties. \\
 
Het concept is ontstaan uit de noodzaak om real-time interacties mogelijk te maken, wat cruciaal is in diverse toepassingen
\cite{Yang2019EdgeDBAE}. \\

Voordelen van Database on Edge omvatten lokale gegevensverwerking, verminderde latency en verbeterde prestaties,
 waardoor het geschikt is voor toepassingen waar snelle respons cruciaal is.

Echter, zoals bij elke technologische benadering, zijn er uitdagingen verbonden aan Database on Edge,
 waaronder het beheer van diverse databases over verschillende apparaten,
 \\ optimalisatie van resourcegebruik op Edge Devices, en het waarborgen van consistente prestaties. \\

De optimalisatie van Database on Edge is een actief onderzoeksgebied 
 en omvat het verkennen van efficiënte algoritmen voor gegevensverwerking,
  verbeteringen in gegevensopslag, en het beheer van gedistribueerde databases om de prestaties te maximaliseren.

\newpage

%---------- Methodologie ------------------------------------------------------

\section{Methodologie}%
\label{sec:methodologie}

Om de onderzoeksvraag te beantwoorden, gaan we gebruik maken van een combinatie van literatuurstudie en een grondige vergelijkende studie. \\
 
Specifieke methoden, zoals requirements-analyse en experimenten, worden toegepast.

We gaan geen enquetes uitvoeren, omdat we de voorkeur geven aan een meer persoonlijke aanpak. \\
 
De technische diepte van deze bachelorproef wordt benadrukt,
 inclusief de beschrijving van gebruikte tools zoals hardware, software en diensten. \\

Een gedetailleerde tijdsplanning wordt opgesteld om de duur van elke onderzoeksfase en de concrete deliverables te definiëren.
Deze tijdsplanning wordt opgesteld in de vorm van een Gantt chart. Deze zal u vinden op de laatste pagina van dit voorstel.




%---------- Verwachte resultaten ----------------------------------------------

\section{Verwachte resultaten en Conclusie}%
\label{sec:verwachte_resultaten}

De verwachte resultaten omvatten concrete aanbevelingen voor werknemers met betrekking tot de optimalisatie
van gegevensopslag in Edge-omgevingen. \\

Grafieken met verwachte conclusies zullen worden gepresenteerd,
 de eerste grafiek zal gaan over het vergelijken van ongeoptimaliseerde en geoptimaliseerde systemen. \\

Met als x-as de tijd en op de y-as de latency per milliseconde. \\

In de conclusie wordt benadrukt hoe de verwachte resultaten bijdragen aan het oplossen van de gedefinieerde probleemsituatie. \\

De conclusie zal gebaseerd zijn op de verzamelde gegevens en resultaten van het onderzoek.

\newpage
\section{Tijdsplanning Methodologie}%
\label{sec:tijdsplanning}
\begin{ganttchart}[
  hgrid,
  vgrid={*{6}{draw=none},dotted},
  x unit=0.14cm,
  y unit title=0.6cm,
  y unit chart=0.6cm,
  title height=0.5,
  bar height=0.5,
  bar/.append style={
      fill=blue!50,
  },
  time slot format=isodate,
  time slot unit=day,
  link bulge=5,
  bar label node/.style={font=\tiny},
  calendar week text={W\currentweek}
]{2024-02-01}{2024-05-30}

\gantttitlecalendar{year, month=shortname, week} \\

\ganttbar{Indienen verbeterd voorstel}{2024-02-16}{2024-02-16} \\
\ganttlinkedbar{Bronnen zoeken}{2024-02-17}{2024-02-20} \\
\ganttlinkedbar{Bronnen uitschrijven}{2024-02-21}{2024-03-25} \\
\ganttlinkedbar{Analyse Methodologie}{2024-03-26}{2024-04-05} \\
\ganttlinkedbar{Conclusie Methodologie}{2024-04-06}{2024-05-20} \\
\ganttlinkedbar{Bachelorproef herlezen}{2024-05-21}{2024-05-25} \\
\ganttlinkedbar{Bachelorproef indienen}{2024-05-25}{2024-05-26} \\

\end{ganttchart}

%```mermaid
%flowchart TD
    %A[Indienen verbeterd voorstel] --> B[Bronnen zoeken];
    %B --> C[Bronnen uitschrijven];
    %C --> D[Analyse Methodologie];
    %D --> E[Conclusie Methodologie];
    %E --> F[Bachelorproef herlezen];
    %F --> G[Bachelorproef indienen];
%```


%%---------- Andere bijlagen --------------------------------------------------
% TODO: Voeg hier eventuele andere bijlagen toe. Bv. als je deze BP voor de
% tweede keer indient, een overzicht van de verbeteringen t.o.v. het origineel.
%\input{...}

%%---------- Backmatter, referentielijst ---------------------------------------

\backmatter{}

\setlength\bibitemsep{2pt} %% Add Some space between the bibliograpy entries
\printbibliography[heading=bibintoc]

\end{document}
