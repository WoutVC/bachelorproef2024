%%=============================================================================
%% Methodologie
%%=============================================================================

\chapter{\IfLanguageName{dutch}{Methodologie}{Methodology}}%
\label{ch:methodologie}

%% TODO: In dit hoofstuk geef je een korte toelichting over hoe je te werk bent
%% gegaan. Verdeel je onderzoek in grote fasen, en licht in elke fase toe wat
%% de doelstelling was, welke deliverables daar uit gekomen zijn, en welke
%% onderzoeksmethoden je daarbij toegepast hebt. Verantwoord waarom je
%% op deze manier te werk gegaan bent.
%% 
%% Voorbeelden van zulke fasen zijn: literatuurstudie, opstellen van een
%% requirements-analyse, opstellen long-list (bij vergelijkende studie),
%% selectie van geschikte tools (bij vergelijkende studie, "short-list"),
%% opzetten testopstelling/PoC, uitvoeren testen en verzamelen
%% van resultaten, analyse van resultaten, ...
%%
%% !!!!! LET OP !!!!!
%%
%% Het is uitdrukkelijk NIET de bedoeling dat je het grootste deel van de corpus
%% van je bachelorproef in dit hoofstuk verwerkt! Dit hoofdstuk is eerder een
%% kort overzicht van je plan van aanpak.
%%
%% Maak voor elke fase (behalve het literatuuronderzoek) een NIEUW HOOFDSTUK aan
%% en geef het een gepaste titel.

In dit hoofdstuk wordt de aanpak van het onderzoek toegelicht. 
 Om de onderzoeksvraag te beantwoorden wordt een combinatie van literatuurstudie, een vergelijkende analyse en een proof of concept toegepast.

De proof of concept richt zich op het implementeren en testen van databases die gebruikmaken van verschillende partitioneringstechnieken.
 De prestaties van deze databases worden gemeten op basis van gedefinieerde criteria zoals snelheid, schaalbaarheid en betrouwbaarheid in een Edge-omgeving.

De technische diepte van deze bachelorproef wordt benadrukt,
 inclusief de beschrijving van gebruikte tools zoals hardware, software en diensten. \\

Een gedetailleerde tijdsplanning wordt opgesteld om de duur van elke onderzoeksfase en de concrete deliverables te definiëren.
Deze tijdsplanning wordt opgesteld in de vorm van een Gantt chart en een flowchart. Deze zal u vinden op de laatste pagina van dit voorstel.

Het onderzoek verloopt in vier fases, die elk gericht zijn op bepaalde doelstellingen en methoden. \\

\section*{Fase 1: Literatuurstudie}

\textbf{Doelstelling:}  
Inzicht krijgen in data-partitioneringstechnieken, vereisten van edge-databases en toepassingen in sensornetwerken.

\textbf{Aanpak:}
\begin{itemize}
    \item Analyse van wetenschappelijke literatuur en officiële documentatie.
    \item Identificatie van functionele en niet-functionele vereisten voor databases in edge-omgevingen.
    \item Opstellen van een longlist van mogelijke databases en technieken.
\end{itemize}

\textbf{Beantwoorde deelvragen:}
\begin{itemize}
    \item \textbf{Welke uitdagingen ontstaan bij het verwerken van grote hoeveelheden IoT-data in een centrale cloudomgeving?} \\ 
    Literatuur toont aan waar cloudverwerking tekortschiet in netwerken met veel sensoren.
    \item \textbf{Wat zijn de functionele vereisten van databases voor het verwerken van real-time sensordata in een edge-context met beperkte connectiviteit?} \\
    Deze vraag wordt beantwoord aan de hand van een literatuurstudie naar bestaande databaseoplossingen die geschikt zijn voor gebruik in edge-omgevingen met sensornetwerken.
\end{itemize}

\textbf{Deliverable:} Een overzicht van de belangrijkste inzichten omtrent partitionering, edge computing en real-time monitoring. \\
\textbf{Deadline:} 21/11/2024

\section*{Fase 2: Selectie en vergelijking van databases}

\textbf{Doelstelling:}  
Databasetechnologieën identificeren die geschikt zijn voor edge-omgevingen en deze theoretisch vergelijken op basis van relevante criteria.

\textbf{Aanpak:}
\begin{itemize}
    \item Opstellen van evaluatiecriteria zoals latentie, fouttolerantie, schaalbaarheid en gebruiksgemak.
    \item Selectie van drie geschikte databases: Cassandra, MongoDB en TimescaleDB.
    \item Vergelijkende analyse van deze databases aan de hand van documentatie en bestaande benchmarks.
\end{itemize}

\textbf{Beantwoorde deelvraag:}
\begin{itemize}
    \item \textbf{Welke van de geselecteerde edge-databases leveren de beste prestaties op vlak van latentie, fouttolerantie en schaalbaarheid binnen de HoGent-use case?} \\
    De vergelijkende studie onderzoekt de prestaties van de databases en hun mogelijke geschiktheid voor de use case.
\end{itemize}

\textbf{Deliverable:} Een gemotiveerde shortlist met drie geschikte edge-databases. \\
\textbf{Deadline:} 05/12/2024

\section*{Fase 3: Proof of Concept}

\textbf{Doelstelling:}  
De geselecteerde databases implementeren in een gesimuleerde edge-omgeving en hun prestaties testen met voorbeeldgegevens.

\textbf{Aanpak:}
\begin{itemize}
    \item Opzetten van een testomgeving met Docker Compose die een edge-scenario nabootst.
    \item Genereren van testdata die sensorwaardes simuleert (CO₂, druk, temperatuur om de 5 seconden).
    \item Implementeren van list-based en range-based partitionering.
    \item Testen op latency, schaalbaarheid, foutafhandeling en beschikbaarheid.
    \item Verzamelen en analyseren van meetgegevens via scripts.
\end{itemize}

\textbf{Beantwoorde deelvragen:}
\begin{itemize}
    \item \textbf{Hoe beïnvloeden verschillende data-partitioneringstechnieken de latentie bij dataverwerking in een Edge Computing-omgeving?} \\
    Door latency te meten in verschillende opstellingen met verschillende technieken.
    \item \textbf{Wat is het effect van verschillende data-partitioneringstechnieken op de netwerkbelasting bij het verwerken van real-time sensordata in een gedistribueerd sensornetwerk?} \\
    Netwerkverkeer wordt gemeten tijdens het testen van de technieken.
    \item \textbf{Wat is de impact van partitioneringsstrategieën op de schaalbaarheid van edge-databases bij toenemende datavolumes?} \\
    Door het effect van groeiende datasets te analyseren bij elke techniek.
\end{itemize}

\textbf{Deliverable:} Een rapport met testresultaten, grafieken en evaluatie per techniek en database. \\
\textbf{Deadline:} 24/12/2024

\section*{Fase 4: Analyse en evaluatie}

\textbf{Doelstelling:}  
Conclusies trekken over welke database en partitioneringstechniek de beste prestaties leveren in deze edge-context.

\textbf{Aanpak:}
\begin{itemize}
    \item Analyse van testresultaten uit de PoC.
    \item Vergelijking op basis van prestatiecriteria.
    \item Formuleren van aanbevelingen voor toekomstige toepassingen.
    \item Bepalen van een totaalscore per database aan de hand van genormaliseerde prestatie-indicatoren, waarbij elke indicator gewogen wordt op basis van de relevantie voor edge computing.
\end{itemize}

\textbf{Beantwoorde deelvraag:}
\begin{itemize}
    \item \textbf{Welke van de geselecteerde edge-databases leveren de beste prestaties op vlak van latentie, fouttolerantie en schaalbaarheid binnen de HoGent-use case?} \\
    De analyse van de testresultaten toont welke oplossing het best past bij de gesimuleerde toepassing.
\end{itemize}

\textbf{Deliverable:} Een eindrapport met visuele samenvattingen, conclusies en aanbevelingen. Inclusief een rangschikking van alle databases op basis van genormaliseerde prestatiecriteria en bijbehorende wegingsfactoren. \\
\textbf{Deadline:} 05/01/2025