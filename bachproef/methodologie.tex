%%=============================================================================
%% Methodologie
%%=============================================================================

\chapter{\IfLanguageName{dutch}{Methodologie}{Methodology}}%
\label{ch:methodologie}

%% TODO: In dit hoofstuk geef je een korte toelichting over hoe je te werk bent
%% gegaan. Verdeel je onderzoek in grote fasen, en licht in elke fase toe wat
%% de doelstelling was, welke deliverables daar uit gekomen zijn, en welke
%% onderzoeksmethoden je daarbij toegepast hebt. Verantwoord waarom je
%% op deze manier te werk gegaan bent.
%% 
%% Voorbeelden van zulke fasen zijn: literatuurstudie, opstellen van een
%% requirements-analyse, opstellen long-list (bij vergelijkende studie),
%% selectie van geschikte tools (bij vergelijkende studie, "short-list"),
%% opzetten testopstelling/PoC, uitvoeren testen en verzamelen
%% van resultaten, analyse van resultaten, ...
%%
%% !!!!! LET OP !!!!!
%%
%% Het is uitdrukkelijk NIET de bedoeling dat je het grootste deel van de corpus
%% van je bachelorproef in dit hoofstuk verwerkt! Dit hoofdstuk is eerder een
%% kort overzicht van je plan van aanpak.
%%
%% Maak voor elke fase (behalve het literatuuronderzoek) een NIEUW HOOFDSTUK aan
%% en geef het een gepaste titel.

Om de onderzoeksvraag te beantwoorden, gaan we gebruik maken van een combinatie van literatuurstudie en een grondige vergelijkende studie. \\
 
Specifieke methoden, zoals requirements-analyse en experimenten, worden toegepast.

Bij deze evaluatie zullen we de edge databases beoordelen op basis van een set van vooraf gedefinieerde criteria, zoals:

\begin{enumerate}
    \item \textbf{Prestaties}:
    \begin{itemize}
        \item Insertsnelheid
        \item Queryprestaties
        \item Latency
    \end{itemize}
    
    \item \textbf{Schaalbaarheid}:
    \begin{itemize}
        \item Horizontale schaalbaarheid
        \item Verticale schaalbaarheid
    \end{itemize}
    
    \item \textbf{Betrouwbaarheid}:
    \begin{itemize}
        \item Fault tolerance
        \item Data consistency
    \end{itemize}
    
    \item \textbf{Geheugen- en resourcegebruik}:
    \begin{itemize}
        \item Geheugenverbruik
        \item CPU-gebruik
    \end{itemize}
    
    \item \textbf{Functionaliteiten}:
    \begin{itemize}
        \item Ondersteuning voor tijdreeksgegevens
        \item Ondersteuning voor edge computing
    \end{itemize}
    
    \item \textbf{Ondersteuning}:
    \begin{itemize}
        \item Documentatie
    \end{itemize}
\end{enumerate}

We zullen geen enquetes uitvoeren, omdat we de voorkeur geven aan een meer persoonlijke aanpak.
 
De technische diepte van deze bachelorproef wordt benadrukt,
 inclusief de beschrijving van gebruikte tools zoals hardware, software en diensten.

Een gedetailleerde tijdsplanning wordt opgesteld om de duur van elke onderzoeksfase en de concrete deliverables te definiëren.
Deze tijdsplanning wordt opgesteld in de vorm van een Gantt chart. Deze zal u vinden op de laatste pagina van dit voorstel.


