%%=============================================================================
%% Methodologie
%%=============================================================================

\chapter{\IfLanguageName{dutch}{Methodologie}{Methodology}}%
\label{ch:methodologie}

%% TODO: In dit hoofstuk geef je een korte toelichting over hoe je te werk bent
%% gegaan. Verdeel je onderzoek in grote fasen, en licht in elke fase toe wat
%% de doelstelling was, welke deliverables daar uit gekomen zijn, en welke
%% onderzoeksmethoden je daarbij toegepast hebt. Verantwoord waarom je
%% op deze manier te werk gegaan bent.
%% 
%% Voorbeelden van zulke fasen zijn: literatuurstudie, opstellen van een
%% requirements-analyse, opstellen long-list (bij vergelijkende studie),
%% selectie van geschikte tools (bij vergelijkende studie, "short-list"),
%% opzetten testopstelling/PoC, uitvoeren testen en verzamelen
%% van resultaten, analyse van resultaten, ...
%%
%% !!!!! LET OP !!!!!
%%
%% Het is uitdrukkelijk NIET de bedoeling dat je het grootste deel van de corpus
%% van je bachelorproef in dit hoofstuk verwerkt! Dit hoofdstuk is eerder een
%% kort overzicht van je plan van aanpak.
%%
%% Maak voor elke fase (behalve het literatuuronderzoek) een NIEUW HOOFDSTUK aan
%% en geef het een gepaste titel.

In dit hoofdstuk wordt de aanpak van het onderzoek toegelicht.

Om de onderzoeksvraag te beantwoorden wordt er een combinatie van literatuurstudie,
 een vergelijkende studie en een proof of concept toegepast waarin de databasetechnologieën ObjectBox en SQLite getest worden. \\

De literatuurstudie zal inzicht bieden in de huidige stand van zaken met betrekking tot data-partitioneringstechnieken in Edge-omgevingen. \\
 
De proof of concept zal zich richten op het implementeren en testen van databases die gebruik maken van consistent hashing en range-based partitionering.
 De prestaties van deze databases worden gemeten op basis van gedefinieerde criteria zoals snelheid, schaalbaarheid, en betrouwbaarheid in een Edge-omgeving.

De technische diepte van deze bachelorproef wordt benadrukt,
 inclusief de beschrijving van gebruikte tools zoals hardware, software en diensten. \\

Een gedetailleerde tijdsplanning wordt opgesteld om de duur van elke onderzoeksfase en de concrete deliverables te definiëren.
Deze tijdsplanning wordt opgesteld in de vorm van een Gantt chart en een flowchart. Deze zal u vinden op de laatste pagina van dit voorstel.

Het onderzoek verloopt in vier fases, die elk gericht zijn op bepaalde doelstellingen en methoden. \\

\textbf{Fase 1: Literatuuronderzoek}\newline\newline
    \textbf{Doelstelling:} Het opzoeken van bestaande studies met betrekking tot Edge Computing en data-partitioneringstechnieken zoals consistent hashing en range-based partitionering.\newline\newline
    \textbf{Aanpak:}
    \begin{itemize}
      \item Identificeren van relevante studies en trends in Edge Computing en data-partitionering.
      \item De belangrijkste bevindingen die betrekking tot de impact van partitionering op databaseprestatie hebben documenteren.
      \item Opstellen van een long list van mogelijke databasetechnologieën.
    \end{itemize}
    \textbf{Beantwoorde deelvragen}:
    \begin{itemize}
    \item \textbf{Hoe verhouden verschillende data-partitioneringstechnieken zich in termen van latentie in een Edge-omgeving?} \\
      De literatuurstudie zal inzicht geven in welke partitioneringstechnieken momenteel worden toegepast en wat hun verwachte invloed is op latentie.
    \item \textbf{Wat is de invloed van data-partitioneringstechnieken op netwerkprestaties in gedistribueerde Edge-netwerken?} \\
      Dit wordt onderzocht door bestaande studies en metingen die zich richten op de impact van partitionering op bandbreedte en netwerkprestaties.
    \item \textbf{Hoe beïnvloeden deze technieken de schaalbaarheid van databases in een dynamische Edge-omgeving?} \\
      Door literatuur te analyseren over schaalbaarheidsuitdagingen in Edge Computing.
    \end{itemize}
    \textbf{Resultaat, deliverable(s):} Een overzicht van relevante literatuur met betrekking tot Edge Computing en data-partitioneringstechnieken.\newline\newline
    \textbf{Deadline:} 21/11/2024\newline\newline
\textbf{Fase 2: Shortlist en Vergelijkende Studie}\newline\newline
    \textbf{Doelstelling:} De databasetechnologieën uit de longlist beoordelen en een shortlist samenstellen, gevolgd door een gedetailleerde vergelijking van de geselecteerde technologieën op basis van hun prestaties.\newline\newline
    \textbf{Aanpak:}
    \begin{itemize}
        \item De long list van databasetechnologieën evalueren op basis van criteria zoals prestaties, schaalbaarheid, kosten en ondersteuning.
        \begin{itemize}
          \item \textbf{Prestaties}: Bepalen aan de hand van responstijd en doorvoersnelheid (het aantal verzoeken dat de database per seconde kan verwerken).
          \item \textbf{Schaalbaarheid}: Testen hoe databases omgaan met toenemende datahoeveelheden en gebruikersbelasting, en hoe ze presteren bij het toevoegen of verwijderen van nodes.
        \end{itemize}
        \item Opstellen van een short list van de meest geschikte databasetechnologieën.
        \item Gedetailleerde vergelijkingen uitvoeren tussen de databasetechnologieën op basis van gedefinieerde criteria.

        \item Gebruik van identieke datasets en testomgevingen om consistentie in de resultaten te garanderen.
        \item De gegevens in tabellen en grafieken presenteren om de vergelijkingen visueel te maken.
        \item De sterke en zwakke punten van elke databasetechnologie identificeren en documenteren.
    \end{itemize}
    \textbf{Beantwoorde deelvragen}:
    \begin{itemize}
    \item \textbf{Welke bestaande Edge-databasetechnologieën leveren, gemeten aan de hand van latentie, fouttolerantie en schaalbaarheid, de beste prestaties in combinatie met de onderzochte partitioneringstechnieken binnen de context van Optis?} \\
      De vergelijkende studie onderzoekt de prestaties van de databases en hun mogelijke geschiktheid voor Optis.
    \end{itemize}
    \textbf{Resultaat, deliverable(s):} Tabellen en grafieken die de prestaties van databases tonen op basis van de gebruikte partitioneringstechniek.\newline\newline
    \textbf{Deadline:} 05/12/2024\newline\newline
\textbf{Fase 3: PoC implementeren}\newline\newline
\textbf{Doelstelling:} Het implementeren en testen van meerdere geselecteerde databasetechnologieën uit de shortlist om hun praktische prestaties te vergelijken in een gesimuleerde edge-omgeving.\newline\newline
\textbf{Aanpak:}
    \begin{itemize}
        \item Selecteren van meerdere databasetechnologieën (zoals Cassandra, MongoDB en TimescaleDB) uit de shortlist, gebaseerd op relevantie voor edge-toepassingen.
        \item Opzetten van een gesimuleerde edge-omgeving met behulp van Docker Compose om consistente testomstandigheden te garanderen.
        \item Gebruik van identieke datasets, zoals gesynthetiseerde IoT-sensordata, om een eerlijke vergelijking tussen de databases mogelijk te maken.
        \item Installeren en configureren van de benodigde software en tools, inclusief scriptbestanden voor dataloading en simulatie.
        \item Meten van prestaties op basis van vooraf gedefinieerde criteria, waaronder latentie, schaalbaarheid, throughput en fouttolerantie.
        \item Het analyseren en visualiseren van de resultaten met behulp van grafieken en tabellen, om duidelijke vergelijkingen te presenteren.
    \end{itemize}
    \textbf{Beantwoorde deelvragen}:
    \begin{itemize}
    \item \textbf{Hoe verhouden verschillende data-partitioneringstechnieken zich in termen van latentie in een Edge-omgeving?} \\
      Door latency-tests uit te voeren op de verschillende partitioneringstechnieken in een gecontroleerde omgeving.
    \item \textbf{Wat is de invloed van data-partitioneringstechnieken op netwerkprestaties in gedistribueerde Edge-netwerken?} \\
      Door de netwerkprestaties te meten in de PoC, kunnen we concrete data verzamelen om de efficiëntie van elke techniek te bepalen.
    \item \textbf{Hoe beïnvloeden deze technieken de schaalbaarheid van databases in een dynamische Edge-omgeving?} \\
      Door schaalbaarheid te testen onder verschillende netwerkcondities en datasetgroottes.
    \end{itemize}
    \textbf{Resultaat, deliverable(s):} Een gedetailleerd PoC-rapport met testresultaten met grafieken en tabellen die de prestaties van de verschillende databasetechnologieën tonen.\newline\newline
    \textbf{Deadline:} 24/12/2024\newline\newline
\textbf{Fase 4: Analyse en evaluatie}\newline\newline
    \textbf{Doelstelling:} De resultaten uit de PoC analyseren en evalueren om conclusies en aanbevelingen te formuleren.\newline\newline
    \textbf{Aanpak:}
    \begin{itemize}
        \item De verzamelde gegevens en testresultaten grondig analyseren.
        \item Vergelijkingen maken op basis van de gedefinieerde criteria en prestaties.
        \item De belangrijkste bevindingen samenvatten en aanbevelingen formuleren voor de beste databasetechnologieën op basis van de analyse.\newline
    \end{itemize}
    \textbf{Beantwoorde deelvragen}:
    \begin{itemize}
    \item \textbf{Welke bestaande Edge-databasetechnologieën leveren, gemeten aan de hand van latentie, fouttolerantie en schaalbaarheid, de beste prestaties in combinatie met de onderzochte partitioneringstechnieken binnen de context van Optis?} \\
      De evaluatie maakt duidelijk welke technologieën daadwerkelijk presteren in de Edge-omgeving, rekening houdend met de specifieke behoeftes van Optis.
    \end{itemize}
    \textbf{Resultaat, deliverable(s):} Een uitgebreid evaluatierapport met conclusies, aanbevelingen en een overzicht van de belangrijkste bevindingen uit de PoC.\newline\newline
    \textbf{Deadline:} 05/01/2025\newline\newline

