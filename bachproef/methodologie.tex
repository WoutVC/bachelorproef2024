%%=============================================================================
%% Methodologie
%%=============================================================================

\chapter{\IfLanguageName{dutch}{Methodologie}{Methodology}}%
\label{ch:methodologie}

%% TODO: In dit hoofstuk geef je een korte toelichting over hoe je te werk bent
%% gegaan. Verdeel je onderzoek in grote fasen, en licht in elke fase toe wat
%% de doelstelling was, welke deliverables daar uit gekomen zijn, en welke
%% onderzoeksmethoden je daarbij toegepast hebt. Verantwoord waarom je
%% op deze manier te werk gegaan bent.
%% 
%% Voorbeelden van zulke fasen zijn: literatuurstudie, opstellen van een
%% requirements-analyse, opstellen long-list (bij vergelijkende studie),
%% selectie van geschikte tools (bij vergelijkende studie, "short-list"),
%% opzetten testopstelling/PoC, uitvoeren testen en verzamelen
%% van resultaten, analyse van resultaten, ...
%%
%% !!!!! LET OP !!!!!
%%
%% Het is uitdrukkelijk NIET de bedoeling dat je het grootste deel van de corpus
%% van je bachelorproef in dit hoofstuk verwerkt! Dit hoofdstuk is eerder een
%% kort overzicht van je plan van aanpak.
%%
%% Maak voor elke fase (behalve het literatuuronderzoek) een NIEUW HOOFDSTUK aan
%% en geef het een gepaste titel.

In dit hoofdstuk wordt de onderzoeksaanpak toegelicht. 
Om de centrale onderzoeksvraag te beantwoorden is een combinatie van literatuurstudie, een vergelijkende analyse en een proof of concept toegepast. 
De proof of concept richtte zich op het implementeren en testen van databases die gebruikmaken van verschillende partitioneringstechnieken. 
De prestaties van deze databases zijn gemeten op basis van vooraf gedefinieerde criteria zoals latentie, throughput, schaalbaarheid, fouttolerantie en consistentie in een gesimuleerde edge-omgeving.

Het onderzoek is uitgevoerd in vier opeenvolgende fasen, waarbij elke fase specifieke doelstellingen, methoden en resultaten omvatte.

\section*{Fase 1: Literatuurstudie}

\textbf{Doelstelling:}  
Verkrijgen van inzicht in data-partitioneringstechnieken, de vereisten van edge-databases en de toepassing daarvan in sensornetwerken.

\textbf{Aanpak:}
\begin{itemize}
    \item Analyse van wetenschappelijke literatuur en officiële documentatie.
    \item Identificatie van functionele en niet-functionele vereisten voor databases in edge-omgevingen.
    \item Opstellen van een longlist van mogelijke databases en technieken.
\end{itemize}

\textbf{Beantwoorde deelvragen:}
\begin{itemize}
    \item \emph{Welke uitdagingen ontstaan bij het verwerken van grote hoeveelheden IoT-data in een centrale cloudomgeving?}
    \item \emph{Wat zijn de functionele vereisten van databases voor het verwerken van real-time sensordata in een edge-context met beperkte connectiviteit?}
\end{itemize}

\textbf{Deliverable:} Overzicht van de belangrijkste inzichten omtrent partitionering, edge computing en real-time monitoring.\\

\section*{Fase 2: Selectie en vergelijking van databases}

\textbf{Doelstelling:}  
Identificeren van databasetechnologieën die geschikt zijn voor edge-omgevingen en deze theoretisch vergelijken op basis van relevante criteria.

\textbf{Aanpak:}
\begin{itemize}
    \item Vastleggen van evaluatiecriteria zoals latentie, fouttolerantie, schaalbaarheid en beheerbaarheid.
    \item Selectie van drie geschikte databases: Cassandra, MongoDB en TimescaleDB.
    \item Uitvoeren van een vergelijkende analyse aan de hand van documentatie en bestaande benchmarks.
\end{itemize}

\textbf{Beantwoorde deelvraag:}
\begin{itemize}
    \item \emph{Welke van de geselecteerde edge-databases leveren de beste prestaties op vlak van latentie, fouttolerantie en schaalbaarheid binnen de HoGent-use case?}
\end{itemize}

\textbf{Deliverable:} Een gemotiveerde shortlist met drie geschikte edge-databases.\\

\section*{Fase 3: Proof of Concept}

\textbf{Doelstelling:}  
Implementeren van de geselecteerde databases in een gesimuleerde edge-omgeving en meten van hun prestaties met gesimuleerde sensordata.

\textbf{Aanpak:}
\begin{itemize}
    \item Opzetten van een testomgeving met Docker Compose die zowel centrale als edge-scenario's nabootst.
    \item Genereren van real-time testdata (CO₂, luchtdruk, temperatuur) met een interval van 5 seconden.
    \item Toepassen van list-based en range-based partitionering.
    \item Uitvoeren van tests op latentie, throughput, schaalbaarheid, fouttolerantie en consistentie.
    \item Verzamelen en analyseren van meetgegevens via geautomatiseerde scripts.
\end{itemize}

\textbf{Beantwoorde deelvragen:}
\begin{itemize}
    \item \emph{Hoe beïnvloeden verschillende data-partitioneringstechnieken de latentie bij dataverwerking in een Edge Computing-omgeving?}
    \item \emph{Wat is het effect van verschillende data-partitioneringstechnieken op de netwerkbelasting bij het verwerken van real-time sensordata in een gedistribueerd sensornetwerk?}
    \item \emph{Wat is de impact van partitioneringsstrategieën op de schaalbaarheid van edge-databases bij toenemende datavolumes?}
\end{itemize}

\textbf{Deliverable:} Rapport met testresultaten, grafieken en evaluatie per techniek en database.\\

\section*{Fase 4: Analyse en evaluatie}

\textbf{Doelstelling:}  
Bepalen welke database en partitioneringstechniek de beste prestaties leveren in de gesimuleerde edge-context.

\textbf{Aanpak:}
\begin{itemize}
    \item Analyse van de testresultaten uit de PoC.
    \item Vergelijking op basis van vastgelegde prestatiecriteria.
    \item Formuleren van aanbevelingen voor toekomstige toepassingen.
    \item Rangschikken van de databases aan de hand van genormaliseerde prestatie-indicatoren met wegingsfactoren op basis van relevantie voor edge computing.
    \item Voor iedere test zijn meerdere runs uitgevoerd; resultaten zijn gemiddeld, genormaliseerd en gewogen volgens de methodiek beschreven in het resultatenhoofdstuk.
\end{itemize}

\textbf{Beantwoorde deelvraag:}
\begin{itemize}
    \item \emph{Welke van de geselecteerde edge-databases leveren de beste prestaties op vlak van latentie, fouttolerantie en schaalbaarheid binnen de HoGent-use case?}
\end{itemize}

\textbf{Deliverable:} Eindrapport met visuele samenvattingen, conclusies, aanbevelingen en rangschikking van de databases.\\