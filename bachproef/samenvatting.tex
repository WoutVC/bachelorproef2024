%%=============================================================================
%% Samenvatting
%%=============================================================================

% TODO: De "abstract" of samenvatting is een kernachtige (~ 1 blz. voor een
% thesis) synthese van het document.
%
% Een goede abstract biedt een kernachtig antwoord op volgende vragen:
%
% 1. Waarover gaat de bachelorproef?
% 2. Waarom heb je er over geschreven?
% 3. Hoe heb je het onderzoek uitgevoerd?
% 4. Wat waren de resultaten? Wat blijkt uit je onderzoek?
% 5. Wat betekenen je resultaten? Wat is de relevantie voor het werkveld?
%
% Daarom bestaat een abstract uit volgende componenten:
%
% - inleiding + kaderen thema
% - probleemstelling
% - (centrale) onderzoeksvraag
% - onderzoeksdoelstelling
% - methodologie
% - resultaten (beperk tot de belangrijkste, relevant voor de onderzoeksvraag)
% - conclusies, aanbevelingen, beperkingen
%
% LET OP! Een samenvatting is GEEN voorwoord!

%%---------- Nederlandse samenvatting -----------------------------------------
%
% TODO: Als je je bachelorproef in het Engels schrijft, moet je eerst een
% Nederlandse samenvatting invoegen. Haal daarvoor onderstaande code uit
% commentaar.
% Wie zijn bachelorproef in het Nederlands schrijft, kan dit negeren, de inhoud
% wordt niet in het document ingevoegd.

\IfLanguageName{english}{%
\selectlanguage{dutch}
\chapter*{Samenvatting}
\lipsum[1-4]
\selectlanguage{english}
}{}

%%---------- Samenvatting -----------------------------------------------------
% De samenvatting in de hoofdtaal van het document

\chapter*{\IfLanguageName{dutch}{Samenvatting}{Abstract}}

Deze bachelorproef richt zich op de invloed van Edge Computing en \\ data-partitioneringstechnieken op de prestaties van gedistribueerde databases bij het verwerken van real-time sensordata. De focus ligt op een fictieve HoGent-use case, waarbij elk lokaal is uitgerust met een IoT-sensor die gegevens meet zoals CO$_2$, temperatuur en luchtdruk.

Het doel is om praktische inzichten te formuleren over hoe Edge Computing en verschillende partitioneringstechnieken kunnen bijdragen aan een efficiënte verwerking en opslag van sensordata. Hiervoor wordt een Proof of Concept ontwikkeld waarin centrale en edge-databases zoals Cassandra, MongoDB en TimescaleDB, worden geëvalueerd op belangrijke prestatie-indicatoren zoals latentie, throughput, fouttolerantie en schaalbaarheid.

Dit onderzoek helpt bij het begrijpen van de toepasbaarheid van Edge Computing en datapartitionering in gedistribueerde IoT-omgevingen en levert praktische aanbevelingen voor de implementatie van lokale dataverwerking via edge-nodes.

Om een grondige basis te leggen, werd een literatuurstudie uitgevoerd naar bestaande Edge Database-technologieën, relevante data-partitioneringstechnieken en de uitdagingen bij real-time dataverwerking in gedistribueerde netwerken.

Op basis van deze inzichten werd de Proof of Concept gerealiseerd. In deze PoC wordt een testomgeving gebruikt om de impact van verschillende partitioneringstechnieken op databaseprestaties te analyseren, waarbij zowel centrale als edge-opstellingen worden meegenomen.

Het onderzoek levert een combinatie van theoretische en praktische bijdragen, waaronder een overzicht van de huidige ontwikkelingen binnen Edge Database-architecturen en aanbevelingen voor de optimale toepassing van Edge Computing en datapartitionering in real-time sensornetwerken.