%%=============================================================================
%% Samenvatting
%%=============================================================================

% TODO: De "abstract" of samenvatting is een kernachtige (~ 1 blz. voor een
% thesis) synthese van het document.
%
% Een goede abstract biedt een kernachtig antwoord op volgende vragen:
%
% 1. Waarover gaat de bachelorproef?
% 2. Waarom heb je er over geschreven?
% 3. Hoe heb je het onderzoek uitgevoerd?
% 4. Wat waren de resultaten? Wat blijkt uit je onderzoek?
% 5. Wat betekenen je resultaten? Wat is de relevantie voor het werkveld?
%
% Daarom bestaat een abstract uit volgende componenten:
%
% - inleiding + kaderen thema
% - probleemstelling
% - (centrale) onderzoeksvraag
% - onderzoeksdoelstelling
% - methodologie
% - resultaten (beperk tot de belangrijkste, relevant voor de onderzoeksvraag)
% - conclusies, aanbevelingen, beperkingen
%
% LET OP! Een samenvatting is GEEN voorwoord!

%%---------- Nederlandse samenvatting -----------------------------------------
%
% TODO: Als je je bachelorproef in het Engels schrijft, moet je eerst een
% Nederlandse samenvatting invoegen. Haal daarvoor onderstaande code uit
% commentaar.
% Wie zijn bachelorproef in het Nederlands schrijft, kan dit negeren, de inhoud
% wordt niet in het document ingevoegd.

\IfLanguageName{english}{%
\selectlanguage{dutch}
\chapter*{Samenvatting}
\lipsum[1-4]
\selectlanguage{english}
}{}

%%---------- Samenvatting -----------------------------------------------------
% De samenvatting in de hoofdtaal van het document

\chapter*{\IfLanguageName{dutch}{Samenvatting}{Abstract}}

Deze bachelorproef richt zich op de invloed van data-partitioneringstechnieken en edge-databases op de prestaties van een Edge Computing-omgeving, met een specifieke focus op de toepassing bij Optis.

Het doel is om praktische aanbevelingen te formuleren voor Optis door de impact van verschillende partitioneringstechnieken op databaseprestaties te analyseren binnen een simulatie. 

Dit onderzoek helpt bij het begrijpen van de toepasbaarheid van partitionering in schaalbare omgevingen en levert praktische inzichten die door Optis kunnen worden toegepast.

Om een grondige basis te leggen, werd een uitgebreide literatuurstudie uitgevoerd naar bestaande Edge Database-technologieën en data-partitioneringstechnieken.

Op basis van deze inzichten werd er een Proof of Concept aangemaakt. In deze PoC gebruikten we een testomgeving waarin verschillende databases zoals Cassandra, MongoDB en TimescaleDB werden geëvalueerd op belangrijke prestaties zoals latentie, throughput, schaalbaarheid en fouttolerantie.

Het onderzoek biedt een combinatie van theoretische en praktische bijdragen, waaronder een duidelijk overzicht van de huidige ontwikkelingen binnen Edge Database-architecturen met bijzondere aandacht voor de vereisten van organisaties zoals Optis.

De resultaten dragen bij aan de verbetering en toepassing van Edge Computing-technologieën in verschillende sectoren.