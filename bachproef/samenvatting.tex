%%=============================================================================
%% Samenvatting
%%=============================================================================

% TODO: De "abstract" of samenvatting is een kernachtige (~ 1 blz. voor een
% thesis) synthese van het document.
%
% Een goede abstract biedt een kernachtig antwoord op volgende vragen:
%
% 1. Waarover gaat de bachelorproef?
% 2. Waarom heb je er over geschreven?
% 3. Hoe heb je het onderzoek uitgevoerd?
% 4. Wat waren de resultaten? Wat blijkt uit je onderzoek?
% 5. Wat betekenen je resultaten? Wat is de relevantie voor het werkveld?
%
% Daarom bestaat een abstract uit volgende componenten:
%
% - inleiding + kaderen thema
% - probleemstelling
% - (centrale) onderzoeksvraag
% - onderzoeksdoelstelling
% - methodologie
% - resultaten (beperk tot de belangrijkste, relevant voor de onderzoeksvraag)
% - conclusies, aanbevelingen, beperkingen
%
% LET OP! Een samenvatting is GEEN voorwoord!

%%---------- Nederlandse samenvatting -----------------------------------------
%
% TODO: Als je je bachelorproef in het Engels schrijft, moet je eerst een
% Nederlandse samenvatting invoegen. Haal daarvoor onderstaande code uit
% commentaar.
% Wie zijn bachelorproef in het Nederlands schrijft, kan dit negeren, de inhoud
% wordt niet in het document ingevoegd.

\IfLanguageName{english}{%
\selectlanguage{dutch}
\chapter*{Samenvatting}
\lipsum[1-4]
\selectlanguage{english}
}{}

%%---------- Samenvatting -----------------------------------------------------
% De samenvatting in de hoofdtaal van het document

\chapter*{\IfLanguageName{dutch}{Samenvatting}{Abstract}}

Mijn bachelorproef richt zich op het onderzoek en begrijpen van de complexiteit rond gegevensopslag in Edge-omgevingen, met een specifieke focus op Edge Database-architecturen. Het doel van mijn onderzoek is niet alleen om praktische aanbevelingen te formuleren, maar eerder om nieuwe inzichten en kennis op het gebied van gegevensopslag in Edge-omgevingen te verkennen.

Ik heb een grondige literatuurstudie uitgevoerd om een stevige basis te leggen voor mijn begrip van Edge Database-architecturen en gerelateerde concepten. Vervolgens heb ik interviews gehouden met experts uit de industrie om diepgaande inzichten en praktijkervaringen te verkrijgen met betrekking tot het optimaliseren van gegevensopslag in Edge-omgevingen.

Na het verzamelen van informatie uit de literatuurstudie en interviews, ben ik overgegaan tot een grondige analyse van de gegevens. Hierbij heb ik patronen geïdentificeerd, vergelijkingen gemaakt tussen verschillende benaderingen en belangrijke inzichten opgedaan die mijn verdere onderzoek hebben gestuurd.

Mijn onderzoek biedt een gebalanceerde mix van theoretische kennis en praktische inzichten, wat heeft geleid tot een diepgaand begrip van Edge Database-architecturen en hun optimalisatiestrategieën.

