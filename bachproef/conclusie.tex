%%=============================================================================
%% Conclusie
%%=============================================================================

\chapter{Conclusie}%
\label{ch:conclusie}

% TODO: Trek een duidelijke conclusie, in de vorm van een antwoord op de
% onderzoeksvra(a)g(en). Wat was jouw bijdrage aan het onderzoeksdomein en
% hoe biedt dit meerwaarde aan het vakgebied/doelgroep? 
% Reflecteer kritisch over het resultaat. In Engelse teksten wordt deze sectie
% ``Discussion'' genoemd. Had je deze uitkomst verwacht? Zijn er zaken die nog
% niet duidelijk zijn?
% Heeft het onderzoek geleid tot nieuwe vragen die uitnodigen tot verder 
%onderzoek?

In dit onderzoek werd nagegaan hoe data-partitioneringstechnieken en edge-databases de prestaties van real-time dataverwerking beïnvloeden binnen een Edge Computing-architectuur. De proof of concept werd uitgevoerd in een gesimuleerd scenario gebaseerd op een HoGent-use case, waarin elk lokaal via een edge-node sensordata verzamelt over \ce{CO2}, temperatuur en luchtdruk.

De resultaten tonen aan dat een edge-gebaseerde aanpak aanzienlijke voordelen biedt ten opzichte van centrale verwerking. Door data lokaal te verwerken en op te slaan, daalt de latentie en wordt het netwerk minder belast. Partitioneringstechnieken bleken bovendien een bepalende factor te zijn voor databaseprestaties op het vlak van snelheid, schaalbaarheid en fouttolerantie.

\section{Vergelijking van databases}

De evaluatie van TimescaleDB, Cassandra en MongoDB in zowel gecentraliseerde als edge-opstellingen leverde de volgende inzichten op:

\begin{itemize}
    \item \textbf{TimescaleDB met range-based partitionering} behaalde de hoogste totaalscore (7.76). Deze configuratie scoorde goed op schaalbaarheid (5.67) en consistentie, en is zeer geschikt voor toepassingen met tijdreeksdata.
    \item \textbf{Cassandra met list-based partitionering} volgde kort daarna (7.74) en behaalde de laagste gemeten latentie (2.82 ms). Dit maakt Cassandra een sterke keuze in omgevingen waar reactiesnelheid en fouttolerantie belangrijk zijn.
    \item \textbf{MongoDB} had goede resultaten voor consistentie en throughput, maar scoorde het laagst op schaalbaarheid (1.95) en eindigde buiten de top drie.
\end{itemize}

\section{Beantwoording van de deelvragen}

\textbf{Welke uitdagingen ontstaan bij het verwerken van grote hoeveelheden IoT-data in een centrale cloudomgeving?} \\
Centrale verwerking zorgt voor hogere latentie, beperkte schaalbaarheid en is kwetsbaar bij netwerkproblemen. Edge Computing kan deze problemen deels oplossen.

\textbf{Wat zijn de functionele vereisten van databases voor het verwerken van real-time sensordata in een edge-context met beperkte connectiviteit?} \\
Databases moeten fouttolerant zijn, snel reageren (lage latentie), goed kunnen schalen, offline blijven werken en eenvoudig te beheren zijn in kleine, lokale systemen.

\textbf{Hoe beïnvloeden verschillende data-partitioneringstechnieken de latentie in een Edge Computing-omgeving?} \\
Cassandra met list-based partitionering had de laagste latentie (2.82 ms), gevolgd door MongoDB met list-based sharding (3.57 ms). De gekozen partitioneringstechniek heeft dus een duidelijke invloed op de reactiesnelheid.

\textbf{Wat is het effect van partitionering op netwerkbelasting en doorvoersnelheid?} \\
TimescaleDB met list-partitionering behaalde de hoogste throughput (4.86 records/s), gevolgd door andere TimescaleDB en MongoDB-configuraties. Zowel list als range-partitionering dragen bij aan een vlotte verwerking van data.

\textbf{Wat is de impact van partitioneringsstrategieën op schaalbaarheid bij toenemende datavolumes?} \\
TimescaleDB met range-partitionering scoorde het hoogst op schaalbaarheid (5.67), gevolgd door Cassandra met list-based partitionering (4.79). Deze technieken zijn dus geschikt voor grotere hoeveelheden data.

\textbf{Welke database biedt de beste prestaties in termen van latentie, fouttolerantie en schaalbaarheid in de HoGent-use case?} \\
De combinatie van goede schaalbaarheid, fouttolerantie en stabiele prestaties maakt TimescaleDB met range-partitionering de beste keuze. Cassandra is een goed alternatief wanneer vooral lage latentie belangrijk is.

\section{Reflectie en Aanbevelingen}

Deze studie bevestigt dat Edge Computing en data-partitionering real-time dataverwerking efficiënter en betrouwbaarder kunnen maken in IoT-toepassingen. De resultaten tonen aan dat databasekeuze en partitioneringstechniek samen gekozen moeten worden, afhankelijk van wat het belangrijkste is: snelheid, schaalbaarheid of fouttolerantie.

\textbf{Aanbevelingen voor verder onderzoek:}
\begin{itemize}
    \item Vergelijking met andere databaseoplossingen zoals Redis, InfluxDB of Apache IoTDB, inclusief hun geschiktheid voor gebruik in edge-omgevingen.
    \item Verkenning van hybride partitioneringsmodellen die automatisch kunnen schakelen op basis van datavolume of gebruik.
    \item Integratie van een Fog Computing-laag tussen edge en cloud, om data eerst lokaal te verwerken of samen te voegen voordat het naar de cloud gaat.
\end{itemize}