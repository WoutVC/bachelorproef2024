%%=============================================================================
%% Conclusie
%%=============================================================================

\chapter{Conclusie}%
\label{ch:conclusie}

% TODO: Trek een duidelijke conclusie, in de vorm van een antwoord op de
% onderzoeksvra(a)g(en). Wat was jouw bijdrage aan het onderzoeksdomein en
% hoe biedt dit meerwaarde aan het vakgebied/doelgroep? 
% Reflecteer kritisch over het resultaat. In Engelse teksten wordt deze sectie
% ``Discussion'' genoemd. Had je deze uitkomst verwacht? Zijn er zaken die nog
% niet duidelijk zijn?
% Heeft het onderzoek geleid tot nieuwe vragen die uitnodigen tot verder 
%onderzoek?
Deze studie onderzocht hoe data-partitioneringstechnieken en edge-databases de prestaties van een Edge Computing-omgeving beïnvloeden.

Ik had verwacht dat de keuze van de data-partitioneringstechniek een aanzienlijke invloed zou hebben op de prestaties van de database in termen van latentie, netwerkprestaties en schaalbaarheid. De resultaten bevestigden dit vermoeden en toonden aan dat de keuze van de partitioneringstechniek meestal een groot verschil maakt in de prestaties van de database.

De resultaten tonen aan dat de keuze van de data-partitioneringstechniek een aanzienlijke invloed heeft op de prestaties van de database in termen van latentie, netwerkprestaties en schaalbaarheid. De vergelijkende studie tussen Cassandra, TimescaleDB en MongoDB toonde aan dat elke database verschillende sterktes en zwaktes heeft, afhankelijk van de vereisten van de toepassing.

De belangrijkste aanbeveling voor Optis is om de technologie te kiezen op basis van prioriteiten:
\begin{itemize}
    \item \textbf{Cassandra} wordt vaak gebruikt voor systemen die schaalbaarheid en robuustheid vereisen. Een typisch voorbeeld van een toepassing van Cassandra is in grote e-commerceplatforms die wereldwijd opereren. Deze platforms verwerken dagelijks enorme hoeveelheden gebruikersdata en transactie-informatie. Cassandra biedt de mogelijkheid om data over meerdere datacenters te repliceren, wat zorgt voor hoge beschikbaarheid en het vermogen om horizontaal te schalen, zonder prestatieverlies, zelfs tijdens piekuren. Dit maakt Cassandra bijzonder geschikt voor toepassingen die grote hoeveelheden gestructureerde en ongestructureerde gegevens vereisen en waarbij continuïteit van de dienstverlening essentieel is.
    \item \textbf{TimescaleDB} is een uitstekende keuze voor toepassingen die real-time gegevensverwerking en lage latentie vereisen. Een voorbeeld hiervan is een applicatie voor het monitoren van Internet of Things (IoT)-apparaten, zoals slimme thermostaten.  
    \item \textbf{MongoDB} biedt flexibiliteit in dataverwerking, vooral wanneer de gegevensstructuur varieert en niet strikt gestructureerd is. Een voorbeeld van het gebruik van MongoDB is een sociale netwerksite waar gebruikers berichten, foto's en video's kunnen delen. Door het documentgebaseerde datamodel van MongoDB kunnen verschillende soorten gegevens, zoals tekst, afbeeldingen en video's, efficiënt worden opgeslagen en opgehaald. Dit maakt MongoDB ideaal voor toepassingen waar de datatypes regelmatig kunnen veranderen en waar snel en flexibel gewerkt moet worden met ongestructureerde gegevens.
\end{itemize}


\section{Beantwoording van de Deelvragen}
\begin{itemize}
    \item \textbf{Hoe verhouden verschillende data-partitioneringstechnieken zich in termen van latentie in een Edge-omgeving?} \\
    Gebaseerd op de resultaten behaalde MongoDB met range-based sharding de laagste latentie (1.37 ms), gevolgd door MongoDB met hash-based sharding (1.4 ms). TimescaleDB met range-based partitionering scoorde 1.61 ms, terwijl Cassandra met range-based partitionering iets hoger scoorde met 1.82 ms. Dit toont aan dat MongoDB over het algemeen betere latentieprestaties levert, terwijl de keuze van de partitioneringstechniek afhankelijk is van de queryvereisten.

    \item \textbf{Wat is de invloed van data-partitioneringstechnieken op netwerkprestaties?} \\
    De metingen laten zien dat MongoDB met hash-based sharding de laagste netwerklatentie behaalde (54.96 ms), gevolgd door TimescaleDB met list-based partitionering (55.91 ms). Cassandra met range-based partitionering scoorde 59.06 ms, wat wijst op robuustheid onder verhoogde netwerkbelasting. Dit benadrukt dat de gekozen partitioneringstechniek de netwerkprestaties aanzienlijk beïnvloedt.

    \item \textbf{Hoe beïnvloeden deze technieken de schaalbaarheid van databases in een dynamische Edge-omgeving?} \\
    Cassandra met consistent hashing behaalde de hoogste schaalbaarheidsscore (2.34), gevolgd door Cassandra met range-based partitionering (2.28). TimescaleDB en MongoDB bleken minder schaalbaar met scores van respectievelijk 1.53 en 1.44. Dit toont aan dat consistent hashing de beste optie is voor dynamische uitbreidingen in een Edge-omgeving.

    \item \textbf{Welke bestaande Edge-databasetechnologieën leveren, gemeten aan de hand van latentie, fouttolerantie en schaalbaarheid, de beste prestaties in combinatie met de onderzochte partitioneringstechnieken binnen de context van Optis?} \\
    Cassandra presteerde uitzonderlijk goed in schaalbaarheid en fouttolerantie dankzij consistent hashing, terwijl TimescaleDB uitstekende latentieprestaties liet zien voor tijdgebaseerde gegevens. MongoDB behaalde goede resultaten op het gebied van netwerklatentie en flexibiliteit, maar was minder geschikt voor extreme schaalvergroting. De meest geschikte keuze voor Optis is daarom Cassandra met consistent hashing, vanwege de combinatie van schaalbaarheid, fouttolerantie en robuuste prestaties.
\end{itemize}