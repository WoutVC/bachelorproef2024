%%=============================================================================
%% Conclusie
%%=============================================================================

\chapter{Conclusie}%
\label{ch:conclusie}

% TODO: Trek een duidelijke conclusie, in de vorm van een antwoord op de
% onderzoeksvra(a)g(en). Wat was jouw bijdrage aan het onderzoeksdomein en
% hoe biedt dit meerwaarde aan het vakgebied/doelgroep? 
% Reflecteer kritisch over het resultaat. In Engelse teksten wordt deze sectie
% ``Discussion'' genoemd. Had je deze uitkomst verwacht? Zijn er zaken die nog
% niet duidelijk zijn?
% Heeft het onderzoek geleid tot nieuwe vragen die uitnodigen tot verder 
%onderzoek?

In dit onderzoek werd nagegaan hoe data-partitioneringstechnieken en edge-databases de prestaties van real-time dataverwerking beïnvloeden binnen een Edge Computing-architectuur. De proof of concept werd uitgevoerd in een gesimuleerd scenario gebaseerd op een HoGent-use case, waarin elk lokaal via een edge-node sensordata verzamelt over \ce{CO2}, temperatuur en luchtdruk.

In dit onderzoek werd nagegaan hoe data-partitioneringstechnieken en edge-databases de prestaties van real-time dataverwerking beïnvloeden binnen een Edge Computing-architectuur. De proof of concept werd uitgevoerd in een gesimuleerd scenario gebaseerd op een HoGent-use case, waarin elk lokaal via een edge-node sensordata verzamelt over \ce{CO2}, temperatuur en luchtdruk.

De resultaten tonen aan dat een edge-gebaseerde aanpak aanzienlijke voordelen biedt ten opzichte van centrale verwerking. Door data lokaal te verwerken en op te slaan, daalt de latentie en wordt het netwerk minder belast. Partitioneringstechnieken bleken bovendien een bepalende factor te zijn voor databaseprestaties op het vlak van snelheid, schaalbaarheid en fouttolerantie.

\section{Vergelijking van databases}

De evaluatie van TimescaleDB, Cassandra en MongoDB in zowel gecentraliseerde als edge-opstellingen leverde de volgende inzichten op:

\begin{itemize}
    \item \textbf{MongoDB Centralized [Gebouw A]} behaalde de hoogste totaalscore (6.95). Deze configuratie scoorde sterk op fouttolerantie en consistentie, waardoor ze in de centrale context de beste algemene prestaties liet zien.
    \item \textbf{MongoDB (List-Based Sharding) [Gebouw B]} volgde kort daarna (6.84). Vooral de hoge throughput en goede fouttolerantie maken dit een sterke keuze voor real-time verwerking in een edge-omgeving.
    \item \textbf{TimescaleDB Centralized [Gebouw A]} eindigde op de derde plaats (6.70). Dankzij de hoge schaalbaarheid en stabiele prestaties blijft TimescaleDB een aantrekkelijke optie in centrale omgevingen.
\end{itemize}

Cassandra onderscheidde zich door de laagste latentie, maar haalde minder goede totaalscores in vergelijking met MongoDB en TimescaleDB. Dit maakt Cassandra vooral interessant in situaties waar reactiesnelheid belangrijker is dan algemene balans.

\section{Beantwoording van de deelvragen}

\textbf{Welke uitdagingen ontstaan bij het verwerken van grote hoeveelheden IoT-data in een centrale cloudomgeving?} \\  
Centrale verwerking zorgt voor hogere latentie, beperkte schaalbaarheid en is kwetsbaar bij netwerkproblemen. Edge Computing kan deze problemen deels oplossen.

\textbf{Wat zijn de functionele vereisten van databases voor het verwerken van real-time sensordata in een edge-context met beperkte connectiviteit?} \\  
Databases moeten fouttolerant zijn, snel reageren (lage latentie), goed kunnen schalen, offline blijven werken en eenvoudig te beheren zijn in kleine, lokale systemen.

\textbf{Hoe beïnvloeden verschillende data-partitioneringstechnieken de latentie in een Edge Computing-omgeving?} \\  
Cassandra met list-based partitionering had de laagste latentie (4.53 ms), gevolgd door Cassandra met range-based partitionering (4.43 ms) en MongoDB met list-based sharding (4.88 ms). Partitionering heeft dus een merkbare invloed op reactiesnelheid.

\textbf{Wat is het effect van partitionering op netwerkbelasting en doorvoersnelheid?} \\  
MongoDB met list-based sharding behaalde de hoogste throughput (4.91 records/s). Dit bevestigt dat partitionering een belangrijke rol speelt bij het efficiënt verwerken van continue datastromen.

\textbf{Wat is de impact van partitioneringsstrategieën op schaalbaarheid bij toenemende datavolumes?} \\  
TimescaleDB bleek het sterkst te scoren op schaalbaarheid (50.93 ms in de centrale opstelling). Partitionering bij edge-databases leverde eveneens stabiele prestaties, maar centrale TimescaleDB presteerde het best in verticale schaalbaarheid.

\textbf{Welke database biedt de beste prestaties in termen van latentie, fouttolerantie en schaalbaarheid in de HoGent-use case?} \\  
De combinatie van fouttolerantie en consistente prestaties maakt MongoDB de beste keuze in zowel centrale als edge-context. TimescaleDB is vooral sterk in schaalbaarheid, terwijl Cassandra zich onderscheidt met de laagste latentie.

\section{Reflectie en Aanbevelingen}

Deze studie bevestigt dat Edge Computing en data-partitionering real-time dataverwerking efficiënter en betrouwbaarder kunnen maken in IoT-toepassingen. De resultaten tonen aan dat databasekeuze en partitioneringstechniek samen gekozen moeten worden, afhankelijk van wat het belangrijkste is: snelheid, schaalbaarheid of fouttolerantie.

\textbf{Aanbevelingen voor verder onderzoek:}
\begin{itemize}
    \item Vergelijking met andere databaseoplossingen zoals Redis, InfluxDB of Apache IoTDB, inclusief hun geschiktheid voor gebruik in edge-omgevingen.
    \item Verkenning van hybride partitioneringsmodellen die automatisch kunnen schakelen op basis van datavolume of gebruik.
    \item Integratie van een Fog Computing-laag tussen edge en cloud, om data eerst lokaal te verwerken of samen te voegen voordat het naar de cloud gaat.
\end{itemize}