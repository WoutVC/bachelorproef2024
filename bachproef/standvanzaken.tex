\chapter{\IfLanguageName{dutch}{Stand van zaken}{State of the art}}%
\label{ch:stand-van-zaken}

% Tip: Begin elk hoofdstuk met een paragraaf inleiding die beschrijft hoe
% dit hoofdstuk past binnen het geheel van de bachelorproef. Geef in het
% bijzonder aan wat de link is met het vorige en volgende hoofdstuk.

% Pas na deze inleidende paragraaf komt de eerste sectiehoofding.

Dit hoofdstuk bespreekt de huidige kennis en ontwikkelingen met betrekking tot databases in Edge Computing-omgevingen en relevante data-partitioneringstechnieken.
De literatuurstudie focust op de essentie van Edge Computing,
 de functie van databases in deze omgevingen en de specifieke technieken voor partitionering die kunnen helpen bij het verbeteren van de prestaties.
 Dit overzicht biedt de benodigde achtergrondinformatie om de resultaten van het praktische deel van dit onderzoek beter te kunnen interpreteren en in een context te plaatsen.

\section{Definitie van Kernbegrippen}

In dit onderzoek worden verschillende technische termen gebruikt. Hieronder volgt een overzicht van de kernbegrippen die essentieel zijn voor het begrijpen van de literatuurstudie en de daaropvolgende analyse:
 
\begin{itemize}
    \item \textbf{Sharding:} Dit is een techniek waarbij een database wordt opgesplitst in kleinere, beter beheersbare delen die afzonderlijk op verschillende servers worden opgeslagen. Elk deel, of shard, bevat een subset van de gegevens en helpt de schaalbaarheid en prestaties te verbeteren \autocite{Mahmud2020}.

    \item \textbf{Eventual Consistency:} Een consistentiemodel waarbij gegevens in een gedistribueerd systeem na verloop van tijd consistent worden. Dit model wordt vaak toegepast in systemen waar lage latentie belangrijker is dan directe consistentie zoals in IoT-omgevingen \autocite{Cao2020}.

    \item \textbf{Cyclisch (Round-robin):} Bij round-robin partitionering worden records gelijkmatig verdeeld over de beschikbare partities in een herhalend (cyclisch) patroon. Dit betekent dat de data in volgorde over de partities wordt verdeeld en wanneer de laatste partitie bereikt is, wordt het proces opnieuw gestart bij de eerste partitie. Dit zorgt voor een gelijkmatige belasting, maar kan inefficiënt zijn voor gerelateerde queries, omdat de data over verschillende partities is verspreid \autocite{Ponnusamy2024}.

    \item \textbf{Hypertables:} Dit is een concept dat wordt gebruikt in TimescaleDB voor het efficiënt beheren van tijdgebaseerde gegevens. Hypertables zijn eigenlijk tabellen die automatisch worden verdeeld op basis van tijdsintervallen, waardoor gegevens snel kunnen worden opgeslagen en opgehaald. Dit maakt ze uitermate geschikt voor toepassingen die veel tijdsgebonden gegevens genereren zoals IoT-sensoren \autocite{TimescaleDBDocumentation}.

    \item \textbf{ACID-transacties:} ACID staat voor \textit{Atomicity}, \textit{Consistency}, \textit{Isolation} en \textit{Durability}. Deze vier eigenschappen zorgen ervoor dat een database altijd in een consistente en betrouwbare staat blijft, zelfs bij fouten. Dit betekent dat elke bewerking in de database als een ononderbroken geheel wordt uitgevoerd (atomiciteit), gegevens op een consistente manier worden opgeslagen (consistentie), gelijktijdige bewerkingen van verschillende gebruikers geen conflicten veroorzaken (isolatie) en eenmaal opgeslagen gegevens niet verloren gaan (duurzaamheid). Dit is essentieel voor systemen die met kritieke gegevens werken \autocite{Cao2020}.

    \item \textbf{MVCC (Multi-Version Concurrency Control):} Dit is een techniek die ervoor zorgt dat meerdere gebruikers gelijktijdig gegevens kunnen bewerken zonder elkaar in de weg te zitten. MVCC maakt gebruik van meerdere versies van gegevens, zodat lezers altijd de meest recente versie kunnen zien zonder te wachten op schrijfbewerkingen en schrijvers kunnen hun wijzigingen doorvoeren zonder andere gebruikers te blokkeren. Dit maakt het systeem efficiënter en zorgt voor een betere gelijktijdige toegang tot gegevens \autocite{Wiseso2020PerformanceAnalysis}.

    \item \textbf{Query-optimalisaties:} Dit zijn technieken die worden gebruikt om zoekopdrachten in een database sneller en efficiënter te maken. Bijvoorbeeld door de manier waarop gegevens worden opgehaald te verbeteren of door de database zo in te richten dat zoekopdrachten minder tijd kosten. Dit is belangrijk in systemen waar snel toegang tot grote hoeveelheden data vereist is zoals bij real-time toepassingen \autocite{Gyorodi2015comparative}.
\end{itemize}

\section{Edge Computing: Definitie en Belang}
\label{sec:edge-computing}

Edge Computing is een gedistribueerde IT-architectuur waarbij clientgegevens aan de rand van het netwerk worden verwerkt, in plaats van deze naar een centrale server te sturen\autocite{Shi2016}. Door verwerking dichter bij de bron te brengen, worden latentie verminderd, netwerkverkeer beperkt en de responstijden verbeterd.

Met de toename van het aantal IoT-apparaten die aanzienlijke hoeveelheden data genereren, biedt Edge Computing een efficiëntere oplossing voor gegevensverwerking\autocite{Shi2016}. Waar traditionele cloudmodellen worstelen met bandbreedtebeperkingen en hoge vertragingen, maakt Edge Computing het mogelijk om gegevens lokaal te analyseren, wat essentieel is voor toepassingen zoals real-time monitoring.

Taheri en Deng onderstrepen dat lage latentie een kernvoordeel is van Edge Computing\autocite{Taheri2020}. Door lokaal data te verwerken, wordt de responstijd sterk verlaagd, wat vooral van belang is in toepassingen zoals omgevingsmonitoring en kritieke IoT-systemen.

Een extra voordeel is de verhoogde betrouwbaarheid van gegevensconsistency. Edge-databases zoals ObjectBox gebruiken transactionele consistentie en synchronisatieprotocollen om ervoor te zorgen dat data correct blijft, zelfs bij tijdelijke netwerkstoringen\autocite{Rahmani2018, Taheri2020}.

\section{Dataverwerking in Edge Computing}
\label{sec:dataverwerking-edge}

De verschuiving naar Edge Computing vraagt om databasesystemen die geschikt zijn voor gedistribueerde omgevingen. Traditionele databases zijn vaak geoptimaliseerd voor centrale opslag en vereisen een stabiele netwerkverbinding, wat in een Edge-context niet altijd mogelijk is\autocite{Kleppmann2017}.

Edge-databases zijn specifiek ontworpen om:

\begin{itemize}
    \item Data lokaal te verwerken;
    \item Offline te blijven functioneren en later te synchroniseren;
    \item Horizontaal schaalbaar te zijn in gedistribueerde netwerken;
    \item Hoge fouttolerantie en beschikbaarheid te bieden;
    \item Sterke beveiliging en privacybescherming te realiseren.
\end{itemize}

Door deze eigenschappen kunnen Edge-databases de performantie van systemen aanzienlijk verbeteren, vooral in omgevingen waar continue netwerkconnectiviteit niet gegarandeerd is. Dit maakt Edge-databases geschikt voor toepassingen zoals real-time monitoring, IoT-netwerken en slimme steden.

In tegenstelling tot klassieke databases, die sterk afhankelijk zijn van een centrale server, brengen Edge-databases de gegevensverwerking dichter bij de bron. Hierdoor kunnen ze latency verminderen, de systeemrespons versnellen en operationeel blijven, zelfs bij tijdelijke netwerkstoringen\autocite{Rahmani2018, Taheri2020}.

\section{Vergelijking: Traditionele Databases vs. Edge Databases}
\label{sec:vergelijking-databases}

Hoewel databases zoals Cassandra, MongoDB en TimescaleDB oorspronkelijk ook in centrale omgevingen worden gebruikt, zijn ze door hun architectuur bijzonder geschikt voor Edge Computing\autocite{Rahmani2018}.

Belangrijke verschillen tussen traditionele databases en edge-databases zijn:

\begin{itemize}
    \item \textbf{Locatie van gegevensverwerking:} Bij traditionele databases wordt de verwerking gecentraliseerd in datacenters, wat betekent dat gegevens vaak over lange afstanden naar het datacenter moeten worden gestuurd voor verwerking. Bij edge-databases wordt de verwerking lokaal uitgevoerd, dicht bij de databron, waardoor de latentie wordt verminderd.
    \item \textbf{Afhankelijkheid van netwerkconnectiviteit:} Traditionele databases zijn sterk afhankelijk van een constante netwerkverbinding tussen de client en het datacenter. Bij verlies van netwerkconnectiviteit kunnen traditionele databases vaak geen toegang tot gegevens bieden. Edge-databases daarentegen kunnen offline functioneren en later synchroniseren wanneer de verbinding wordt hersteld.
    \item \textbf{Schaalbaarheid:} Traditionele databases kunnen schaalbaarheid bereiken door verticaal te schalen (het upgraden van servers) of door complexe systemen zoals sharding te implementeren. Edge-databases zijn ontworpen om horizontaal uit te breiden door nieuwe nodes toe te voegen zonder de werking van het systeem te onderbreken, wat vooral nuttig is in gedistribueerde omgevingen.
    \item \textbf{Fouttolerantie en beschikbaarheid:} Traditionele databases vertrouwen vaak op replicatie en clustering om de beschikbaarheid te waarborgen, maar zijn afhankelijk van netwerkverbindingen en centrale knooppunten. Als het centrale knooppunt uitvalt, kan de toegang tot de gegevens worden vestoord. Edge-databases bieden replicatie en automatische failover om gegevensbeschikbaarheid te garanderen, zelfs bij netwerk- of knooppuntstoringen, doordat ze lokaal en gedistribueerd werken.
\end{itemize}

Door deze verschillen zijn edge-databases beter geschikt voor real-time toepassingen in gedistribueerde netwerken zoals IoT-systemen en slimme steden.
\subsection{Toepassingen van Edge Computing}

Een belangrijke toepassing van Edge Computing is te vinden in de gezondheidszorg, waar het kan bijdragen aan het verbeteren van de efficiëntie en effectiviteit van medische systemen. Rahmani onderzoekt in zijn artikel de rol van Edge Computing in de context van gezondheidszorg en IoT \autocite{Rahmani2018}.
 Ze presenteren een benadering waarbij zogenaamde 'smart e-health gateways' worden gebruikt om medische gegevens te verwerken en analyseren aan de rand van het netwerk, waardoor een snellere en meer betrouwbare gegevensverwerking mogelijk wordt gemaakt.
Volgens hem kan de toepassing van Edge Computing in de gezondheidszorg aanzienlijk bijdragen aan de efficiëntie van gegevensverwerking door het verminderen van de latentie en het verbeteren van de betrouwbaarheid van medische systemen. 
 Door de verwerking van gegevens dichter bij de bron zoals medische sensoren en draagbare apparaten, kunnen deze systemen sneller reageren op kritieke situaties en real-time feedback bieden aan zorgverleners.
Deze benadering helpt niet alleen bij het verbeteren van de responstijd en nauwkeurigheid van medische toepassingen, maar ook bij het verminderen van de belasting op centrale datacenters en netwerken \autocite{Rahmani2018}.
 
Het artikel benadrukt ook het potentieel van Edge Computing om bij te dragen aan de ontwikkeling van slimme gezondheidsoplossingen door middel van fog computing-technieken. Fog computing, een uitbreiding van Edge Computing, verwerkt en bewaart data tussen IoT-apparaten en de cloud. Dit zorgt voor een meer gedistribueerde en schaalbare benadering van gegevensverwerking in de gezondheidszorg en biedt verbeterde ondersteuning voor real-time monitoring, analyse en besluitvorming in medische omgevingen \autocite{Rahmani2018}.
 
Het inzicht van Rahmani onderstreept het belang van Edge Computing voor de gezondheidszorgsector en toont aan hoe deze technologie kan bijdragen aan het verbeteren van medische diensten door middel van geavanceerde gegevensverwerking aan de rand van het netwerk.
 Deze toepassing van Edge Computing maakt het mogelijk om gezondheidsgegevens efficiënter te verwerken en sneller te reageren op de behoeften van patiënten, wat cruciaal is voor het succes van moderne e-healthoplossingen \autocite{Rahmani2018}.
 
\paragraph{Slimme steden}  
Edge Computing speelt een sleutelrol in de ontwikkeling van slimme steden door het mogelijk te maken om IoT-gegevens lokaal te verwerken, wat leidt tot snellere besluitvorming en efficiënter gebruik van middelen. Volgens Khan et al. biedt Edge Computing een oplossing voor de beperkingen van cloudgebaseerde systemen zoals hoge latentie en bandbreedteproblemen. Het stelt stedelijke systemen in staat om real-time gegevens te analyseren en snel te reageren op veranderende omstandigheden \autocite{EdgeSmartCities2023}.
 
Een concreet voorbeeld hiervan is het gebruik van Edge Computing voor energiebeheer in slimme steden. Ali et al. beschrijven hoe intelligente edge-apparaten worden ingezet om de energie-efficiëntie in stedelijke infrastructuren zoals smart grids en slimme gebouwen, te verbeteren. Door gegevens lokaal te verwerken, kunnen energienetwerken beter worden afgestemd op de vraag en kunnen energiebronnen effectiever worden beheerd \autocite{EnergyManagement2023}.
 
Daarnaast benadrukt Khan et al. dat Edge Computing bijdraagt aan betere mobiliteitsoplossingen door verkeersbeheer te ondersteunen. Edge-apparaten langs wegen en kruispunten kunnen verkeersgegevens in real-time analyseren, wat leidt tot een dynamische aanpassing van verkeerslichten en een vermindering van verkeersopstoppingen. Dit bevordert niet alleen de doorstroming, maar vermindert ook de CO₂-uitstoot in stedelijke gebieden \autocite{EdgeSmartCities2023}.
 
Deze toepassingen tonen aan dat Edge Computing de basis legt voor duurzamere en beter beheersbare stedelijke omgevingen. Het stelt slimme steden in staat om flexibeler en responsiever te zijn, terwijl het tegelijkertijd de belasting op centrale datacenters vermindert en operationele kosten verlaagt.

\section{Data-partitioneringstechnieken: Overzicht en Analyse}

Data-partitionering is een essentiële techniek in gedistribueerde databasesystemen, vooral in Edge Computing-omgevingen. Partitionering helpt bij het verdelen van data over meerdere nodes (servers), wat kan leiden tot betere prestaties, schaalbaarheid en beschikbaarheid \autocite{Karger1997}.
Deze sectie geeft eerst een kort overzicht van veelgebruikte partitioneringstechnieken. Daarna volgt een uitgebreidere bespreking van \textbf{range-based} en \textbf{list-based} partitionering, aangezien deze methodes het meest relevant zijn voor real-time verwerking van sensordata in een Edge Monitoring System.

\subsection{Overzicht van partitioneringstechnieken}

Hieronder volgt een beknopt overzicht van de belangrijkste vormen van data-partitionering:

\begin{itemize}
    \item \textbf{Range-based partitionering}: Gegevens worden verdeeld op basis van reeksen, zoals tijdstippen of numerieke waarden. Deze methode is efficiënt voor tijdgebaseerde opvragingen, maar kan leiden tot een ongelijke verdeling (data skew) als sommige reeksen veel drukker zijn dan andere.
    
    \item \textbf{List-based partitionering}: Data wordt toegewezen aan partities op basis van vaste, vooraf bepaalde waarden (bijvoorbeeld sensornamen of types). Deze methode is geschikt voor gegevens die in duidelijke categorieën vallen.
    
    \item \textbf{Hash-based partitionering}: Een hashfunctie bepaalt in welke partitie de data terechtkomt. Dit zorgt voor een evenwichtige verdeling, maar maakt het moeilijker om gegevens op volgorde (zoals tijd) op te vragen.
    
    \item \textbf{Round-robin partitionering}: Data wordt afwisselend verdeeld over de partities, onafhankelijk van de inhoud. Deze eenvoudige methode houdt geen rekening met het soort data of de verdeling ervan.
    
    \item \textbf{Consistent hashing}: Een flexibelere vorm van hash-partitionering die beter werkt in systemen waar servers vaak worden toegevoegd of verwijderd. Deze aanpak is krachtig, maar technisch complexer.
    
    \item \textbf{Subpartitionering}: Hierbij worden meerdere technieken gecombineerd. Een voorbeeld is eerst partitioneren op regio (list) en daarna op tijd (range). Dit verhoogt de controle, maar maakt het systeem ook ingewikkelder.
\end{itemize}

De focus binnen dit onderzoek ligt op \textbf{range-based} en \textbf{list-based partitionering}, aangezien deze technieken het best aansluiten bij de structuur en noden van een real-time Edge Monitoring System. De volgende paragrafen gaan hier dieper op in.

\subsection{Range-based Partitionering}
Range-based partitionering verdeelt data op basis van specifieke waarde-intervallen zoals datums of numerieke bereiken. Elk interval wordt toegewezen aan een specifieke partitie. Bijvoorbeeld, een dataset met verkoopdata kan worden verdeeld in kwartaal-intervallen: Q1 (januari-maart), Q2 (april-juni), enzovoort.

\paragraph{Voordelen:}
\begin{itemize}
    \item Efficiënt voor range queries: Queries die specifieke bereiken zoeken, kunnen snel worden uitgevoerd zonder alle partities te doorzoeken \autocite{Ponnusamy2024, Kleppmann2017}.
    \item Eenvoudige implementatie: Vooral geschikt voor gestructureerde datasets met natuurlijke ordening zoals tijdreeksen.
\end{itemize}

\paragraph{Nadelen:}
\begin{itemize}
    \item Data skew: Ongelijke dataverdeling kan leiden tot overbelaste partities en verminderde prestaties \autocite{Ponnusamy2024}.
    \item Beperkte schaalbaarheid: Het toevoegen van nieuwe datumbereiken kan herpartitionering vereisen, wat complexiteit toevoegt \autocite{Kleppmann2017}.
\end{itemize}

\paragraph{Invloed op latentie} 
Deze techniek maakt het mogelijk om gerelateerde data in dezelfde partitie te clusteren, waardoor range-queries efficiënter kunnen worden uitgevoerd. Dit verkleint de noodzaak om alle partities te doorzoeken, wat de responstijd verbetert, vooral bij tijdgebaseerde gegevens in Edge Computing-omgevingen. Onbalans in de dataverdeling kan echter leiden tot overbelasting van een enkele partitie \autocite{Mahmud2020}.

\paragraph{Invloed op bandbreedte en netwerkprestaties} 
Door alleen de relevante partities te benaderen, minimaliseert range-based partitionering het netwerkverkeer. Dit optimaliseert het bandbreedtegebruik, maar slecht gebalanceerde ranges kunnen een ongelijke belasting van de nodes veroorzaken \autocite{Ponnusamy2024}.

\paragraph{Invloed op schaalbaarheid} 
Het toevoegen van nieuwe datumbereiken vereist vaak handmatige herpartitionering. In TimescaleDB wordt dit probleem gedeeltelijk opgelost door de implementatie van hypertables, die automatische schaalbaarheid bieden \autocite{TimescaleDBDocumentation}.

\paragraph{Toepassingen:}
\begin{itemize}
    \item Financiële sector: Beheer van historische transacties per tijdsperiode.
    \item IoT-toepassingen: Tijdgebaseerde sensorlogs verdelen \autocite{Ponnusamy2024}.
    \item Sensornetwerken: Tijdgebaseerde opslag van temperatuurmetingen van sensoren in gebouwen of landbouwtoepassingen \autocite{Mahmud2020}.
\end{itemize}
 
\subsection{List-based Partitionering}
List-partitionering verdeelt data op basis van specifieke discrete waarden. Elke waarde of groep waarden wordt toegewezen aan een specifieke partitie.
 
\paragraph{Voordelen:}
\begin{itemize}
    \item Flexibel: Geschikt voor categorische datasets met een vooraf bekende structuur \autocite{Mahmud2020}.
    \item Efficiënt voor specifieke queries: Queries gericht op specifieke categorieën kunnen snel worden uitgevoerd.
\end{itemize}
 
\paragraph{Nadelen:}
\begin{itemize}
    \item Data skew: Ongelijke verdeling van categorieën kan leiden tot overbelaste partities \autocite{Mahmud2020}.
    \item Beperkt aanpasbaar: Moeilijk schaalbaar bij dynamische categorieën of frequente veranderingen.
\end{itemize}
 
\paragraph{Invloed op latentie} 
List-based partitionering kan de latentie verminderen door gerichte queries uit te voeren op specifieke categorieën. Omdat elke categorie een eigen partitie heeft, kunnen zoekopdrachten gericht worden uitgevoerd zonder onnodige data door te nemen. In omgevingen met een grote variëteit aan categorieën kan de latentie echter toenemen wanneer meerdere partities tegelijk moeten worden doorzocht \autocite{Ponnusamy2024, Mahmud2020}.
 
\paragraph{Invloed op bandbreedte en netwerkprestaties} 
Door het beperken van query-verkeer tot de partitie die de relevante data bevat, kan list-based partitionering het bandbreedtegebruik optimaliseren. Wanneer de verdeling van data onevenwichtig is, kan echter een ongelijke belasting van het netwerk ontstaan, waarbij overbelaste partities een hoger dataverkeer genereren dan andere \autocite{Ponnusamy2024}.
 
\paragraph{Invloed op schaalbaarheid} 
List-based partitionering is minder geschikt voor dynamische of sterk groeiende datasets. Wanneer nieuwe categorieën worden toegevoegd, vereist dit vaak een herconfiguratie van partities, wat de schaalbaarheid beperkt. In omgevingen met vooraf gedefinieerde en stabiele categorieën blijft deze techniek echter eenvoudig beheersbaar en efficiënt \autocite{Mahmud2020}.
 
\paragraph{Toepassingen:}
\begin{itemize}
    \item Logistiek: Orders verdelen op basis van leveringsregio.
    \item Retail: Klantgegevens opsplitsen per filiaal \autocite{Ponnusamy2024}.
    \item Sensornetwerken: Opsplitsing van gegevens per sensortype (temperatuur, luchtvochtigheid, lichtintensiteit) om gerichte analyse te vergemakkelijken \autocite{Mahmud2020}.
\end{itemize}

\subsection{Horizontale en Verticale Schaling in Edge Computing}

Schaalbaarheid is een essentiële eigenschap van databases in Edge Computing-omgevingen. Naarmate workloads groeien en infrastructuren dynamischer worden, is het belangrijk dat systemen effectief kunnen uitbreiden. Er wordt onderscheid gemaakt tussen twee vormen van schaalbaarheid: horizontale schaling en verticale schaling.

\paragraph{Horizontale Schaling}
Horizontale schaling of scale-out, houdt in dat extra nodes worden toegevoegd aan een gedistribueerd systeem. Dit type schaling wordt veel gebruikt in databases zoals Cassandra en MongoDB, die specifiek zijn ontworpen voor een gedistribueerde architectuur \autocite{Kleppmann2017}.  
Deze schaling biedt voordelen zoals een verhoogde fouttolerantie en betere ondersteuning voor gedistribueerde workloads, bijvoorbeeld bij het verwerken van IoT-sensordata. 
Tegelijkertijd brengt het uitdagingen met zich mee zoals complexere netwerkconfiguraties en afhankelijkheid van geavanceerde partitioneringstechnieken zoals consistent hashing \autocite{Mahmud2020}.

\paragraph{Verticale Schaling}
Verticale schaling of scale-up, betreft het vergroten van de capaciteit van een enkele node door betere hardware in te zetten zoals snellere processors of meer geheugen \autocite{Ponnusamy2024}. 
Dit soort schaling is geschikt voor databases die werken met gecentraliseerde dataopslag zoals TimescaleDB en PostgreSQL. 
Hoewel verticale schaling eenvoudig te implementeren is, wordt het beperkt door de fysieke grenzen van hardware. 
Daarnaast biedt het minder mogelijkheden voor fouttolerantie, omdat de werklast afhankelijk blijft van een enkele server \autocite{Mahmud2020}.

\paragraph{Relevantie voor Edge Computing}
In Edge Computing-omgevingen wordt vaak gekozen voor horizontale schaling vanwege de gedistribueerde structuur van deze architecturen. 
Door data dicht bij de bronnen zoals IoT-apparaten te verwerken kan de belasting over meerdere nodes worden verdeeld. 
Verticale schaling wordt echter toegepast in scenario's waar beperkte schaalvereisten bestaan of wanneer de verwerking lokaal op één node kan worden uitgevoerd \autocite{Kleppmann2017}.

\subsection{Vergelijking van partitioneringstechnieken}
De volgende tabel geeft een overzicht van de besproken technieken volgens relevante criteria voor edge-omgevingen:

\begin{table}[H]
    \centering
    \caption{Overzicht van partitioneringstechnieken voor Edge Computing \autocite{Mahmud2020, Ponnusamy2024, Kleppmann2017}}
    \resizebox{\textwidth}{!}{%
    \begin{tabular}{|c|p{4.2cm}|p{4.5cm}|p{4.5cm}|c|}
    \hline
    \textbf{Techniek} & \textbf{Voordelen} & \textbf{Nadelen} & \textbf{Toepassingen} & \textbf{Complexiteit} \\ \hline

    \textbf{Range-based} & 
    Goed voor tijdgebaseerde data, makkelijk te gebruiken, minder netwerkverkeer bij gerichte zoekopdrachten & 
    Data kan ongelijk verdeeld zijn, toevoegen van nieuwe tijdsintervallen is lastig, schaalt minder goed & 
    Tijdgebaseerde sensorlogs in IoT, kwartaaltransacties in de financiële sector, tijdgebaseerde sensorlogs in IoT zoals temperatuurmetingen in gebouwen of landbouwtoepassingen & 
    Laag \\ \hline

    \textbf{Hash-based} & 
    Verdeelt data gelijkmatig, makkelijk uitbreidbaar, goed voor veranderende data & 
    Moeilijk bij zoekopdrachten over meerdere waarden, hangt af van een goede hashfunctie & 
    Product-ID's verdelen in e-commerce, gebruikersdata verdelen in sociale media, verdeling van data van IoT-sensoren op basis van unieke sensor-ID's voor load balancing & 
    Gemiddeld \\ \hline

    \textbf{List-based} & 
    Handig voor vaste categorieën, snel bij gerichte zoekopdrachten & 
    Minder geschikt voor veranderende data, sommige categorieën kunnen te vol raken & 
    Orders per regio in logistiek, klantgegevens per filiaal in retail, opsplitsing van gegevens per sensortype voor analyse & 
    Laag \\ \hline

    \textbf{Round-robin} & 
    Verdeelt data netjes en eenvoudig, goed bij continue datastroom & 
    Niet goed bij samenhangende data, samengestelde zoekopdrachten zijn traag & 
    Sensorlogdata verdelen, real-time dataverwerking voor streamingtoepassingen & 
    Laag \\ \hline

    \textbf{Consistent hashing} & 
    Werkt goed als nodes veranderen, geschikt voor grote systemen, beperkt herverdeling & 
    Moeilijker om op te zetten, minder geschikt voor geordende zoekopdrachten & 
    Gedistribueerde caching (Memcached), load balancing in grootschalige cloudomgevingen, dynamische verdeling van sensordata in slimme steden met veranderende node-configuraties & 
    Gemiddeld \\ \hline

    \textbf{Subpartitionering} & 
    Meer controle over waar data zit, beter schaalbaar zonder alles te herverdelen & 
    Moeilijk om te beheren, complexe zoekopdrachten kunnen trager zijn & 
    Tijdgebaseerde hypertables in TimescaleDB, regionale verdeling in multi-geo systemen, opslag van tijdsreeksen per regio en per tijdseenheid & 
    Hoog \\ \hline

    \textbf{Hybride partitionering} & 
    Combineert voordelen van andere methodes, flexibel bij complexe data & 
    Moeilijk te configureren en beheren, heeft vaak speciale software nodig & 
    Tijdgebaseerde logging + verdeling in IoT-systemen, analysesegmentatie op regio/accounttype, tijd en locatiegebaseerde verdeling van sensorlogs voor precisielandbouw & 
    Zeer hoog \\ \hline

    \end{tabular}%
    }
\end{table}

\noindent

\section{Databases voor Edge Computing}

In dit hoofdstuk worden vijf databases besproken die relevant zijn voor Edge Computing: PostgreSQL, Cassandra, Redis, TimescaleDB en MongoDB.
 De selectie van deze databases is gebaseerd op functionele en niet-functionele eisen zoals prestaties, schaalbaarheid en fouttolerantie, die cruciaal zijn voor toepassingen zoals de HoGent-use case. 
De gekozen databases zijn open source, gratis beschikbaar en ondersteunen geavanceerde partitioneringstechnieken.

\subsection{PostgreSQL}

PostgreSQL is een krachtige relationele database die bekend staat om zijn uitgebreide ondersteuning voor gestructureerde gegevensopslag en uitgebreide functionaliteit. Het wordt veel gebruikt in zowel traditionele als moderne toepassingen, inclusief gedistribueerde en Edge Computing-omgevingen. PostgreSQL biedt sterke consistentie via ACID-transacties en ondersteunt uitbreidingen zoals CitusDB voor schaalvergroting in gedistribueerde systemen \autocite{Kleppmann2017, PostgreSQLDocumentation}.

\paragraph{Toepassing in sensornetwerken}  
In een sensornetwerk zoals dat van de HoGent-use case kan PostgreSQL worden ingezet voor het opslaan van metingen per lokaal. Door gebruik te maken van range-based partitionering op tijd (bijvoorbeeld per dag of per uur) en list-based partitionering op lokaal of sensortype, kunnen metingen efficiënt worden opgeslagen en opgevraagd. In combinatie met CitusDB biedt PostgreSQL de mogelijkheid om deze gegevens te verdelen over meerdere nodes, wat schaalbaarheid en fouttolerantie ten goede komt. Dit maakt het geschikt voor scenario’s waarbij meerdere edge-devices gelijktijdig data aanleveren en real-time analyse nodig is \autocite{PostgreSQLDocumentation, Kleppmann2017}.

\paragraph{Partitioneringstechnieken}  
PostgreSQL ondersteunt verschillende partitioneringstechnieken, waaronder:
\begin{itemize}
    \item \textbf{List-based partitionering}: Data wordt verdeeld op basis van specifieke waarden of categorieën (bijv. regio's, productcategorieën).
    \item \textbf{Range-based partitionering}: Data wordt verdeeld op basis van sleutelbereiken zoals tijd of numerieke waarden.
    \item \textbf{Hash-based partitionering} (via CitusDB): Data wordt gelijkmatig verdeeld over meerdere servers door een hashfunctie toe te passen.
    \item \textbf{Subpartitionering}: PostgreSQL ondersteunt subpartitionering als een aanvullende techniek, waarbij een partitionering wordt toegepast binnen een bestaande partitie (bijv. een combinatie van range-based en list-based).
\end{itemize}

\begin{table}[H]
    \centering
    \caption{Overzicht van de specificaties van PostgreSQL. \autocite{PostgreSQLDocumentation}}
    \resizebox{\textwidth}{!}{%
    \begin{tabular}{|l|l|}
    \hline
    \textbf{Database} & \textbf{PostgreSQL} \\ \hline
    \textbf{Programmeertalen} & C, ondersteuning voor bindings in meerdere talen (Python, Java, Go) \\ \hline
    \textbf{Gegevenstypen} & Relationele tabellen, JSON, XML, key-value, tijdreeksen \\ \hline
    \textbf{Geheugen- en resourcegebruik} & Flexibel, geoptimaliseerd voor complexe query's \\ \hline
    \textbf{Distributiemodel} & Replicatie en clustering via extensies zoals CitusDB \\ \hline
    \textbf{Schaalbaarheid en prestaties} & Geschikt voor grootschalige systemen, uitbreidbaar via sharding en replicatie \\ \hline
    \textbf{Beheer en onderhoud} & Geavanceerde beheertools zoals pgAdmin en command-line interface \\ \hline
    \textbf{Beschikbaarheid en betrouwbaarheid} & Hoge beschikbaarheid via failover en replicatie \\ \hline
    \textbf{Ondersteuning en documentatie} & Uitgebreide documentatie en actieve wereldwijde community \\ \hline
    \end{tabular}%  
    }
\end{table}

\subsection{Cassandra}

Apache Cassandra is een gedistribueerde NoSQL-database die speciaal is ontworpen voor het verwerken van grote hoeveelheden gestructureerde data over meerdere servers. Het ondersteunt horizontale schaling en partitionering. Cassandra biedt uitstekende prestaties bij invoegoperaties, vooral in gedistribueerde omgevingen. De gedistribueerde architectuur en geoptimaliseerde opslag-engine maken Cassandra geschikt voor systemen met hoge snelheden voor het toevoegen van gegevens. Dit wordt vaak geprezen in Edge-omgevingen waar snelheid en schaalbaarheid cruciaal zijn \autocite{CassandraDocumentation}.

\paragraph{Toepassing in sensornetwerken}  
In sensornetwerken met hoge gegevensvolumes, zoals honderden IoT-apparaten verspreid over een campus, biedt Cassandra een robuuste oplossing. Elke sensor kan gebruikmaken van een unieke ID die via consistent hashing wordt toegewezen aan een node in het cluster. Zo wordt de databelasting gelijkmatig verdeeld en blijft het systeem schaalbaar en fouttolerant. Dankzij de ondersteuning voor offline nodes en automatische rebalancing is Cassandra bijzonder geschikt voor grootschalige, dynamische edge-netwerken \autocite{CassandraDocumentation, Mahmud2020}.

\paragraph{Partitioneringstechnieken}  
Cassandra maakt gebruik van de volgende partitioneringstechnieken:
\begin{itemize}
    \item \textbf{Consistent hashing}: Data wordt dynamisch verdeeld over meerdere nodes op basis van een hashfunctie en alleen de benodigde data wordt herschikt wanneer nodes worden toegevoegd of verwijderd.
    \item \textbf{Range-based partitionering} (via token range-benadering): De data wordt verdeeld op basis van een bereik van tokens (de waarden die door de hashfunctie worden berekend).
\end{itemize}

\begin{table}[H]
    \centering
    \caption{Overzicht van de specificaties van Apache Cassandra. \cite{CassandraDocumentation}}
    \resizebox{\textwidth}{!}{%
    \begin{tabular}{|l|l|}
    \hline
    \textbf{Database} & \textbf{Apache Cassandra} \\ \hline
    \textbf{Programmeertalen} & Java \\ \hline
    \textbf{Gegevenstypen} & Kolomgebaseerd (Column-family) \\ \hline
    \textbf{Geheugen- en resourcegebruik} & Geoptimaliseerd voor gedistribueerde systemen \\ \hline
    \textbf{Distributiemodel} & Gedistribueerde data-opslag, horizontale partitionering \\ \hline
    \textbf{Schaalbaarheid en prestaties} & Uitstekende horizontale schaalbaarheid, geschikt voor big data toepassingen \\ \hline
    \textbf{Beheer en onderhoud} & Vereist configuratie voor gedistribueerde setup \\ \hline
    \textbf{Beschikbaarheid en betrouwbaarheid} & Hoge beschikbaarheid, fouttolerantie \\ \hline
    \textbf{Ondersteuning en documentatie} & Actieve gemeenschap, gedetailleerde documentatie \\ \hline
    \end{tabular}%
    }
\end{table}

\subsection{Redis}

Redis is een in-memory datastore die ook ondersteuning biedt voor partitionering via Redis Cluster. Het biedt snelle toegang en schaalbaarheid, wat het geschikt maakt voor edge computing. Door de in-memory opslagarchitectuur levert Redis uitstekende prestaties bij het uitvoeren van invoegoperaties. Dit maakt het zeer geschikt voor toepassingen die een lage latentie en hoge doorvoer vereisen zoals bij Edge Computing-toepassingen. De snelle toegang tot gegevens maakt Redis bijzonder nuttig voor real-time toepassingen \autocite{RedisDocumentation}.

\paragraph{Toepassing in sensornetwerken}  
Redis kan in sensornetwerken dienen als een tijdelijke opslaglaag dicht bij de edge, ideaal voor real-time verwerking van data met hoge frequentie. Dankzij zijn in-memory opslag en ondersteuning voor hash-based partitionering via Redis Cluster kunnen gegevensstromen van tientallen sensoren snel worden opgeslagen en verdeeld. Dit is met name nuttig voor toepassingen zoals tijdelijke buffering, alert-systemen of data-analyse op korte termijn (bijvoorbeeld laatste 5 minuten CO₂-waarden) \autocite{RedisDocumentation, Mahmud2020}.

\paragraph{Partitioneringstechnieken}  
Redis ondersteunt de volgende partitioneringstechnieken:
\begin{itemize}
    \item \textbf{Hash-based partitionering} (via Redis Cluster): Data wordt verdeeld over meerdere nodes met behulp van een hashfunctie.
    \item \textbf{Range-based partitionering} (mogelijk via Redis Streams): Dit wordt niet als primaire techniek gebruikt in Redis, maar kan wel via specifieke datastructuren zoals Redis Streams voor tijdgebaseerde data.
\end{itemize}

\begin{table}[H]
    \centering
    \caption{Overzicht van de specificaties van Redis. \cite{RedisDocumentation}}
    \resizebox{\textwidth}{!}{%
    \begin{tabular}{|l|l|}
    \hline
    \textbf{Database} & \textbf{Redis} \\ \hline
    \textbf{Programmeertalen} & C, met bindings voor veel talen (Java, Python, etc.) \\ \hline
    \textbf{Gegevenstypen} & Key-Value, String, List, Set, Hash, etc. \\ \hline
    \textbf{Geheugen- en resourcegebruik} & Geoptimaliseerd voor snelle in-memory data-opslag \\ \hline
    \textbf{Distributiemodel} & Redis Cluster biedt sharding voor horizontale schaalbaarheid \\ \hline
    \textbf{Schaalbaarheid en prestaties} & Zeer snel, schaalbaar met Redis Cluster \\ \hline
    \textbf{Beheer en onderhoud} & Makkelijk te configureren, met ingebouwde replicatie \\ \hline
    \textbf{Beschikbaarheid en betrouwbaarheid} & Ondersteunt fouttolerantie met replicatie \\ \hline
    \textbf{Ondersteuning en documentatie} & Actieve gemeenschap, uitgebreide documentatie \\ \hline
    \end{tabular}%  
    }
\end{table}

\subsection{TimescaleDB}

TimescaleDB is een op PostgreSQL gebaseerde database die speciaal is ontworpen voor tijdreeksdata. Het ondersteunt automatische partitionering voor schaalbaarheid en prestaties. TimescaleDB biedt uitstekende prestaties voor query's in tijdreeksdata. Dit maakt het ideaal voor toepassingen waarbij tijdgebaseerde gegevens snel geanalyseerd moeten worden zoals sensorgegevens in Edge Computing-omgevingen. De automatische partitionering via hypertables zorgt ervoor dat de database efficiënt blijft schalen, zelfs bij het verwerken van zeer grote hoeveelheden tijdgebaseerde data \autocite{TimescaleDBDocumentation}.

\paragraph{Toepassing in sensornetwerken}  
TimescaleDB is zeer geschikt voor de opslag van tijdreeksdata die afkomstig is van sensoren in een edge-omgeving. In de HoGent-use case kunnen metingen van temperatuur, CO₂ en druk worden opgeslagen in een hypertable, waarbij range-based partitionering automatisch gebeurt op basis van tijd. Dit maakt TimescaleDB ideaal voor toepassingen zoals historische analyse, trenddetectie of voorspellend onderhoud binnen gebouwen. De onderliggende PostgreSQL-architectuur zorgt voor betrouwbaarheid en ACID-consistentie \autocite{TimescaleDBDocumentation, Kleppmann2017}.

\paragraph{Partitioneringstechnieken}  
TimescaleDB maakt gebruik van de volgende partitioneringstechnieken:
\begin{itemize}
    \item \textbf{Range-based partitionering} (via hypertables): TimescaleDB gebruikt hypertables die automatisch de data partitioneren op basis van tijdsintervallen.
    \item \textbf{List-based partitionering} (indirect via hypertables als er meerdere tijdsreeksen zijn die in verschillende tabellen worden verdeeld): Dit wordt meestal niet gebruikt voor tijdreeksdata, maar kan in sommige gevallen relevant zijn.
\end{itemize}

\begin{table}[H]
    \centering
    \caption{Overzicht van de specificaties van TimescaleDB. \cite{TimescaleDBDocumentation}}
    \resizebox{\textwidth}{!}{%
    \begin{tabular}{|l|l|}
    \hline
    \textbf{Database} & \textbf{TimescaleDB} \\ \hline
    \textbf{Programmeertalen} & SQL (PostgreSQL) \\ \hline
    \textbf{Gegevenstypen} & Tijdreeksdata (Time-series) \\ \hline
    \textbf{Geheugen- en resourcegebruik} & Geoptimaliseerd voor tijdreeksdata \\ \hline
    \textbf{Distributiemodel} & Hypertables en automatische partitionering van tijdreeksdata \\ \hline
    \textbf{Schaalbaarheid en prestaties} & Zeer goed in tijdreeksdata-analyse en schaalbaar \\ \hline
    \textbf{Beheer en onderhoud} & Gebruiksvriendelijke configuratie via PostgreSQL \\ \hline
    \textbf{Beschikbaarheid en betrouwbaarheid} & Fouttolerantie via replicatie \\ \hline
    \textbf{Ondersteuning en documentatie} & Actieve gemeenschap, gedetailleerde documentatie \\ \hline
    \end{tabular}%  
    }
\end{table}

\subsection{MongoDB}

MongoDB is een populaire NoSQL-database die horizontale schaling en partitionering biedt via sharding. Het is ideaal voor gedistribueerde toepassingen, waaronder edge computing. MongoDB biedt flexibiliteit bij het uitvoeren van queries op semi-gestructureerde data. Dit maakt het een goede keuze voor toepassingen met wisselende datavormen in gedistribueerde systemen. Het biedt ook sterke ondersteuning voor update- en verwijderbewerkingen, die dynamische datasets vereisen zoals vaak het geval is in Edge Computing-toepassingen \autocite{MongoDBDocumentation}.

\paragraph{Toepassing in sensornetwerken}  
MongoDB is vooral nuttig in sensornetwerken waarbij de structuur van de gegevens flexibel is of varieert tussen types sensoren. Bijvoorbeeld, sommige IoT-apparaten kunnen extra velden bevatten of onregelmatig rapporteren. Door middel van sharding via een hash- of range-gebaseerde shard key, kan deze variabele data efficiënt worden verdeeld over meerdere nodes. MongoDB is hierdoor inzetbaar voor edge-omgevingen waar semi-gestructureerde of dynamisch evoluerende dataformaten voorkomen \autocite{MongoDBDocumentation, Mahmud2020}.

\paragraph{Partitioneringstechnieken}  
MongoDB ondersteunt de volgende partitioneringstechnieken:
\begin{itemize}
    \item \textbf{Sharding}: MongoDB verdeelt de data horizontaal over meerdere servers via een shard-sleutel, wat zorgt voor een gelijkmatige belasting van de servers.
    \item \textbf{Hash-based partitionering} (via sharding): De data wordt verdeeld op basis van een hash van de shard-sleutel.
    \item \textbf{Range-based partitionering} (via sharding): In sommige gevallen kan MongoDB ook 'range-based sharding' gebruiken, afhankelijk van de shard-sleutel en het type data.
\end{itemize}

\begin{table}[H]
    \centering
    \caption{Overzicht van de specificaties van MongoDB. \cite{MongoDBDocumentation}}
    \resizebox{\textwidth}{!}{%
    \begin{tabular}{|l|l|}
    \hline
    \textbf{Database} & \textbf{MongoDB} \\ \hline
    \textbf{Programmeertalen} & C++, met bindings voor veel talen (Java, Python, etc.) \\ \hline
    \textbf{Gegevenstypen} & Documenten (BSON) \\ \hline
    \textbf{Geheugen- en resourcegebruik} & Geoptimaliseerd voor document-gebaseerde opslag \\ \hline
    \textbf{Distributiemodel} & Sharding voor horizontale schaalbaarheid \\ \hline
    \textbf{Schaalbaarheid en prestaties} & Goed geschikt voor gedistribueerde systemen, hoge prestaties \\ \hline
    \textbf{Beheer en onderhoud} & Eenvoudige configuratie via MongoDB Atlas (cloud-gebaseerd) \\ \hline
    \textbf{Beschikbaarheid en betrouwbaarheid} & Ondersteunt fouttolerantie met replicatie en automatische failover \\ \hline
    \textbf{Ondersteuning en documentatie} & Actieve gemeenschap, uitgebreide documentatie \\ \hline
    \end{tabular}%  
    }
\end{table}

\subsection{Evaluatie van Partitioneringstechnieken in Edge Computing}

Partitioneringstechnieken worden breed toegepast in moderne databases om de unieke uitdagingen van Edge Computing aan te pakken. Hieronder volgt een evaluatie van enkele veelgebruikte databases en de manier waarop zij partitioneringstechnieken benutten om schaalbaarheid, prestaties en efficiëntie te verbeteren in gedistribueerde omgevingen.

\paragraph{PostgreSQL}  
PostgreSQL maakt gebruik van partitioneringstechnieken en sharding via extensies zoals CitusDB, wat het schaalbaar maakt voor gedistribueerde omgevingen zoals Edge Computing \autocite{PostgreSQLDocumentation}. Deze functionaliteiten stellen de database in staat om gegevens te verdelen over meerdere nodes, waardoor de belasting beter wordt gebalanceerd. Dankzij de combinatie van range-based en hash-based partitionering kan PostgreSQL goed omgaan met zowel gestructureerde datasets als dynamische queryvereisten \autocite{Kleppmann2017}. Dit maakt het een veelzijdige keuze voor Edge Computing-omgevingen waarin schaalbaarheid en consistentie belangrijk zijn.

\paragraph{Cassandra}  
Apache Cassandra maakt gebruik van consistent hashing om data gelijkmatig te verdelen over een cluster van nodes \autocite{CassandraDocumentation}. Deze techniek maakt het systeem bijzonder geschikt voor dynamische omgevingen waarin nodes regelmatig worden toegevoegd of verwijderd. Cassandra is ontworpen voor grote datasets en biedt hoge fouttolerantie en uitstekende prestaties, zelfs bij piekbelastingen \autocite{CassandraDocumentation}. Dit maakt het een optimale keuze voor toepassingen waarin horizontale schaalbaarheid en betrouwbaarheid van cruciaal belang zijn.

\paragraph{Redis}  
Redis is een in-memory database die bekend staat om zijn lage latentie en hoge verwerkingssnelheid. Door gebruik te maken van het Redis Cluster-model kan de data horizontaal worden geshard, wat zorgt voor betere schaalbaarheid \autocite{RedisDocumentation}. Hoewel Redis zeer effectief is in toepassingen met strikte eisen voor responstijden, is de schaalbaarheid sterk afhankelijk van de configuratie van Redis Cluster \autocite{RedisDocumentation}. Dit maakt het bijzonder geschikt voor real-time toepassingen zoals caching en messaging in Edge Computing.

\paragraph{TimescaleDB}  
TimescaleDB is een uitbreiding van PostgreSQL die specifiek is ontworpen voor tijdreeksdata. Het gebruikt hypertables om data automatisch te partitioneren op basis van tijd, wat zorgt voor efficiënte opslag en snelle query-afhandeling \autocite{TimescaleDBDocumentation}. Deze aanpak maakt TimescaleDB bijzonder geschikt voor toepassingen die grote hoeveelheden tijdgebaseerde gegevens genereren zoals sensoren in IoT-toepassingen binnen Edge Computing. Daarnaast biedt het sterke ondersteuning voor schaalbaarheid en query-efficiëntie \autocite{TimescaleDBDocumentation}.

\paragraph{MongoDB}  
MongoDB biedt flexibiliteit bij het beheren van semi-gestructureerde data door middel van sharding. Door data te verdelen op basis van een shard-sleutel, kan MongoDB de belasting gelijkmatig spreiden over meerdere nodes \autocite{MongoDBDocumentation}. Hoewel deze aanpak goed werkt voor dynamische datasets, kan de complexiteit toenemen in zeer veranderlijke omgevingen. MongoDB is met name geschikt voor Edge Computing-toepassingen die werken met verschillende datavormen en waar flexibiliteit een belangrijke vereiste is \autocite{MongoDBDocumentation}.

\newpage

\section{Evaluatie van databases binnen Edge Database-architecturen}

    Het ontwerpen van efficiënte gegevensopslag in Edge Database-architecturen vereist een grondige evaluatie van verschillende databaseoplossingen. Deze evaluatie maakt gebruik van functionele en niet-functionele criteria, gestructureerd volgens het MoSCoW-principe. Dit stelt ons in staat om de prestaties van databases te beoordelen op basis van de specifieke eisen die Edge Computing-omgevingen stellen zoals lage latentie, schaalbaarheid en beveiliging.

    \subsection{Criteria voor Evaluatie}
    De evaluatiecriteria zijn onderverdeeld in twee categorieën:

    \textbf{Niet-functionele eisen:}
    \begin{itemize}
        \item \textbf{Geheugengebruik en CPU-belasting (Should):} De database moet efficiënt draaien op edge-devices met beperkte hardware.
        \item \textbf{Doorvoersnelheid (Must):} Ondersteuning voor verwerking van hoge frequenties aan sensordata.
        \item \textbf{Beheer en onderhoud (Could):} Ondersteuning voor eenvoudige configuratie, monitoring en herstel.
        \item \textbf{Gegevensconsistentie (Must):} Zekerheid dat alle nodes over een betrouwbare en coherente dataset beschikken.
        \item \textbf{Prestaties (Must):} Ondersteuning voor snelle lees- en schrijfbewerkingen van sensordata in een near-realtime context. Lage latency is cruciaal voor continue datastromen.
        \item \textbf{Schaalbaarheid (Must):} Mogelijkheid om horizontaal uit te breiden over meerdere nodes zonder prestatiedaling.
        \item \textbf{Fouttolerantie (Must):} Robuustheid bij netwerkstoringen of node-uitval om gegevensverlies te voorkomen.
        \item \textbf{Offline werking en synchronisatie (Should):} Tijdelijke lokale opslag en veilige synchronisatie wanneer verbinding met de cloud hersteld wordt.
        \item \textbf{Lokale verwerking (Must):} Capaciteit om data op het edge-device zelf te verwerken, wat latentie en bandbreedteverbruik reduceert.
    \end{itemize}

    \subsection{Evaluatie van Criteria per Database}
    De onderstaande tabel geeft een overzicht van de scores van vijf databases op de eerder gedefinieerde functionele en niet-functionele criteria. De scores variëren van 1 (minst geschikt) tot 5 (meest geschikt). Deze evaluatie is gebaseerd op literatuuronderzoek en een vergelijking van technische specificaties en toepassingsscenario’s.

    \begin{table}[H]
        \centering
        \caption{Scores van databases op niet-functionele selectiecriteria. \autocite{Mahmud2020, PostgreSQLDocumentation, CassandraDocumentation, RedisDocumentation, TimescaleDBDocumentation, MongoDBDocumentation}}
        \label{tab:database-scoring}
        \resizebox{\textwidth}{!}{%
        \begin{tabular}{|l|l|l|l|l|l|}
        \hline
        \textbf{Criteria}           & \textbf{PostgreSQL} & \textbf{Cassandra} & \textbf{Redis} & \textbf{TimescaleDB} & \textbf{MongoDB} \\ \hline
        
        Prestaties (lage latentie)           & 3/5                 & 5/5                & 5/5            & 4/5                  & 4/5              \\ \hline
        Schaalbaarheid                       & 3/5                 & 5/5                & 4/5            & 4/5                  & 5/5              \\ \hline
        Fouttolerantie                       & 4/5                 & 5/5                & 4/5            & 4/5                  & 5/5              \\ \hline
        Lokale verwerking op edge-nodes      & 3/5                 & 5/5                & 4/5            & 5/5                  & 4/5              \\ \hline
        Offline werking en synchronisatie    & 3/5                 & 5/5                & 3/5            & 4/5                  & 4/5              \\ \hline
        
        Geheugengebruik en CPU-belasting     & 3/5                 & 4/5                & 5/5            & 4/5                  & 4/5              \\ \hline
        Doorvoersnelheid bij hoge frequentie & 4/5                 & 5/5                & 5/5            & 4/5                  & 4/5              \\ \hline
        Onderhoudsgemak                      & 4/5                 & 3/5                & 4/5            & 4/5                  & 3/5              \\ \hline
        Gegevensconsistentie                 & 5/5                 & 4/5                & 3/5            & 5/5                  & 4/5              \\ \hline
        \textbf{Totaalscore}         & \textbf{32/45}      & \textbf{41/45}     & \textbf{37/45} & \textbf{42/45}       & \textbf{41/45}   \\ \hline
        \end{tabular}%
        }
    \end{table}

    De scores in de tabel hierboven zijn toegekend op basis van een analyse van de relevante eigenschappen en prestaties van de databases in de context van Edge Computing. Hieronder volgt een toelichting op de toekenning van deze scores en de criteria die zijn gebruikt om de databases te evalueren.

    %\paragraph{Toelichting bij de scores}
    %Elke score toont aan hoe goed een database voldoet aan het specifieke voorwaarde in een Edge Computing-omgeving.

    %\begin{itemize}
        %\item \textbf{Prestaties (lage latentie)}: Cassandra en Redis scoren 5/5 dankzij hun geoptimaliseerde architectuur voor real-time toepassingen \autocite{Mahmud2020}. PostgreSQL scoort lager (3/5) vanwege hogere overhead bij complexe query’s \autocite{PostgreSQLDocumentation}. TimescaleDB en MongoDB bieden degelijke prestaties, afhankelijk van de workload (4/5) \autocite{TimescaleDBDocumentation, MongoDBDocumentation}.

        %\item \textbf{Schaalbaarheid}: Cassandra en MongoDB behalen 5/5 vanwege van hun uitstekende horizontale schaalbaarheid via sharding \autocite{CassandraDocumentation, MongoDBDocumentation}. Redis en TimescaleDB zijn goed schaalbaar met clustering of hypertables (4/5), terwijl PostgreSQL meer moeite vereist voor schaalvergroting (3/5) \autocite{RedisDocumentation, TimescaleDBDocumentation, PostgreSQLDocumentation}.

        %\item \textbf{Fouttolerantie}: Cassandra, MongoDB en TimescaleDB bieden ingebouwde ondersteuning voor replicatie en automatische failover (5/5) \autocite{CassandraDocumentation, MongoDBDocumentation, TimescaleDBDocumentation}. Redis en PostgreSQL hebben minder geavanceerde foutafhandeling (4/5) \autocite{RedisDocumentation, PostgreSQLDocumentation}.

        %\item \textbf{Lokale verwerking op edge-nodes}: Cassandra en TimescaleDB zijn volledig geoptimaliseerd voor gedistribueerde verwerking op edge-devices (5/5) \autocite{CassandraDocumentation, TimescaleDBDocumentation}. Redis en MongoDB bieden goede ondersteuning, maar vereisen meer configuratie (4/5). PostgreSQL scoort lager (3/5) door zijn centrale ontwerp \autocite{RedisDocumentation, MongoDBDocumentation, PostgreSQLDocumentation}.

        %\item \textbf{Offline werking en synchronisatie}: Cassandra scoort hoog (5/5) dankzij ondersteuning voor tijdelijke node-uitval en synchronisatie. TimescaleDB en MongoDB bieden robuuste replicatie en kunnen offline data bufferen (4/5). Redis biedt beperkte ondersteuning (3/5), terwijl PostgreSQL vaak afhankelijk is van cloudconnectiviteit (3/5) \autocite{CassandraDocumentation, TimescaleDBDocumentation, MongoDBDocumentation, RedisDocumentation, PostgreSQLDocumentation}.

        %\item \textbf{Geheugengebruik en CPU-belasting}: Redis is licht en efficiënt (5/5), ideaal voor edge-devices \autocite{RedisDocumentation}. TimescaleDB, MongoDB en Cassandra scoren goed (4/5), terwijl PostgreSQL door zijn complexere engine iets zwaarder is (3/5) \autocite{TimescaleDBDocumentation, MongoDBDocumentation, CassandraDocumentation, PostgreSQLDocumentation}.

        %\item \textbf{Doorvoersnelheid bij hoge frequentie}: Redis en Cassandra verwerken grote hoeveelheden data zeer efficiënt (5/5) \autocite{Mahmud2020, RedisDocumentation}. PostgreSQL, TimescaleDB en MongoDB behalen 4/5, afhankelijk van configuratie en workload \autocite{PostgreSQLDocumentation, TimescaleDBDocumentation, MongoDBDocumentation}.

        %\item \textbf{Beheer en onderhoud}: PostgreSQL, Redis en TimescaleDB zijn relatief eenvoudig te beheren met bestaande tooling (4/5) \autocite{PostgreSQLDocumentation, RedisDocumentation, TimescaleDBDocumentation}. Cassandra en MongoDB vereisen meer gespecialiseerde kennis en configuratie (3/5) \autocite{CassandraDocumentation, MongoDBDocumentation}.

        %\item \textbf{Gegevensconsistentie}: PostgreSQL en TimescaleDB ondersteunen volledige ACID-transacties (5/5) \autocite{PostgreSQLDocumentation, TimescaleDBDocumentation}. Cassandra en MongoDB gebruiken eventual consistency, wat goed is voor schaalbaarheid maar iets lager scoort (4/5). Redis mist sterke garanties en scoort daarom lager (3/5) \autocite{CassandraDocumentation, MongoDBDocumentation, RedisDocumentation}.
    %\end{itemize}

    \subsection{Motivering voor Uitsluitingen}

    De evaluatie van verschillende databases heeft geleid tot de uitsluiting van een aantal opties die niet voldoen aan de specifieke eisen van Edge Computing. Hieronder worden de redenen voor deze uitsluitingen toegelicht:

    \paragraph{PostgreSQL}
    PostgreSQL is uitgesloten vanwege de beperkte schaalbaarheidsmogelijkheden in gedistribueerde omgevingen. Hoewel extensies zoals CitusDB sharding ondersteunen, blijft de implementatie hiervan complex en tijdrovend. Daarnaast leidt de hogere overhead bij complexe query’s tot suboptimale prestaties in scenario’s waar lage latentie vereist is \autocite{PostgreSQLDocumentation, Kleppmann2017}.

    \paragraph{Redis}
    Redis biedt uitstekende prestaties op het gebied van latentie, maar het ontbreken van uitgebreide ondersteuning voor persistente opslag en fouttolerantie maakt het ongeschikt voor toepassingen die robuustheid vereisen bij netwerkstoringen of hardwarefalen. Bovendien is Redis niet geoptimaliseerd voor de langdurige verwerking van grootschalige datasets \autocite{RedisDocumentation, Mahmud2020}.

    \subsection{Shortlist op Basis van Evaluatie}
    Op basis van de scores en de evaluatie van de criteria wordt een shortlist samengesteld van databases die het meest geschikt zijn voor Edge Computing. Deze selectie houdt rekening met zowel de functionele als niet-functionele vereisten zoals lage latentie, schaalbaarheid, fouttolerantie en compatibiliteit met Edge-omgevingen.

\begin{itemize}
    \item \textbf{Cassandra:} Cassandra scoort hoog op lage latentie (5/5), schaalbaarheid (5/5), fouttolerantie (5/5) en compatibiliteit met Edge Computing (5/5). Deze database maakt gebruik van consistent hashing, wat zorgt voor een gelijkmatige verdeling van data en efficiënte verwerking in gedistribueerde omgevingen \autocite{CassandraDocumentation}.
    \item \textbf{MongoDB:} MongoDB presteert bijzonder goed op lage latentie (4/5), schaalbaarheid (5/5) en fouttolerantie (5/5) dankzij de flexibiliteit van zijn gedistribueerde architectuur en robuuste replicatiemechanismen \autocite{MongoDBDocumentation}. Het biedt uitgebreide ondersteuning voor beveiliging en gebruiksvriendelijkheid, wat het een veelzijdige keuze maakt voor Edge Computing-toepassingen.
    \item \textbf{TimescaleDB:} TimescaleDB onderscheidt zich in IoT-toepassingen en tijdreeksdata dankzij de ondersteuning voor hypertables, waarmee data efficiënt wordt gepartitioneerd op basis van tijd \autocite{TimescaleDBDocumentation}. Met hoge scores op consistentie (5/5), schaalbaarheid (4/5) en compatibiliteit met Edge Computing (5/5), is TimescaleDB een uitstekende keuze voor toepassingen die werken met grote hoeveelheden tijdgebaseerde gegevens.
\end{itemize}

De keuze voor deze databases is gebaseerd op hun prestaties in de meest relevante criteria voor Edge Computing. Cassandra biedt de beste balans tussen prestaties en schaalbaarheid, Redis blinkt uit in snelheid en eenvoud, terwijl TimescaleDB is ideaal voor nichetoepassingen zoals IoT en tijdreeksen.

%\subsection{Databasekeuze voor Edge-omgevingen}
%De keuze van een database speelt een cruciale rol in Edge-omgevingen, waar lage latentie en hoge prestaties van vitaal belang zijn. Relationele databases zoals PostgreSQL bieden gestructureerde gegevensopslag en strikte integriteitsgaranties via mechanismen zoals ACID-transacties en multi-version concurrency control (MVCC). Deze eigenschappen zijn vooral relevant voor toepassingen die consistentie en betrouwbaarheid vereisen zoals financiële toepassingen en realtime monitoringstoepassingen \autocite{Kleppmann2017}.

%Daarentegen zijn edge-specifieke databases zoals Cassandra, TimescaleDB en MongoDB ontworpen met een focus op schaalbaarheid en prestatieoptimalisatie voor gedistribueerde systemen. Dit biedt voordelen zoals:
%\begin{itemize}
    %\item \textbf{Lage latentie:} Gegevens worden lokaal verwerkt, wat resulteert in snelle responstijden, essentieel voor toepassingen zoals IoT-sensoren en autonome systemen \autocite{Taheri2020}.
    %\item \textbf{Synchronisatie bij beperkte connectiviteit:} Data worden lokaal opgeslagen en alleen gesynchroniseerd wanneer een netwerkverbinding beschikbaar is. Dit maakt dergelijke databases geschikt voor omgevingen met intermitterende netwerken \autocite{Rahmani2018}.
    %\item \textbf{Lichtgewicht ontwerp:} Deze databases zijn ontworpen om efficiënt te functioneren op apparaten met beperkte rekenkracht, zoals mobiele toestellen en IoT-apparaten..
%\end{itemize}

%\subsection{Gegevensintegriteit en consistentie}

%In Edge-omgevingen is het behouden van consistente gegevens een uitdaging vanwege de gedistribueerde structuur van deze systemen. Relationele databases zoals PostgreSQL bieden sterke consistentie via ACID-transacties, wat zorgt voor onmiddellijke consistentie, maar dit kan ten koste gaan van prestaties in gedistribueerde systemen. Gedistribueerde databases zoals Cassandra, Redis, TimescaleDB en MongoDB implementeren *eventual consistency*, wat betekent dat gegevens over tijd consistent worden zonder constante netwerkverbindingen. Deze aanpak maakt ze bijzonder geschikt voor dynamische Edge-netwerken waar netwerkverbindingen niet altijd beschikbaar zijn \autocite{Taheri2020, Kleppmann2017, CassandraDocumentation}.

%\subsection{Schaalbaarheid}

%Schaalbaarheid is een kritische factor in Edge Computing, waar het aantal apparaten en de hoeveelheid gegevens exponentieel kan groeien. PostgreSQL ondersteunt schaalvergroting via replicatie en extensies zoals CitusDB, maar vereist aanvullende configuraties voor geavanceerde sharding. Daarentegen bieden Cassandra, Redis, TimescaleDB en MongoDB ingebouwde schaalbare oplossingen, die naadloos nieuwe apparaten en datasets kunnen integreren door gebruik te maken van sharding en replicatie \autocite{CassandraDocumentation, RedisDocumentation, TimescaleDBDocumentation, MongoDBDocumentation}.

%\subsection{Beveiliging en privacy}

%Beveiliging en privacy zijn fundamenteel in Edge-omgevingen. PostgreSQL biedt uitgebreide ondersteuning voor encryptie, zowel tijdens opslag als tijdens transmissie \autocite{Kleppmann2017}. Edge-databases bieden echter extra voordelen door gegevens lokaal te verwerken, wat de noodzaak van transmissie minimaliseert en risico’s op datalekken vermindert. Veel edge-databases ondersteunen encryptie op apparaatniveau en fijnmazige toegangscontroles, wat hen bijzonder geschikt maakt voor privacygevoelige toepassingen zoals gezondheidszorg en slimme steden \autocite{Taheri2020}.

%\section{Shortlist van databases voor Edge Computing}

    %Op basis van de evaluatie van beschikbare databases zijn Cassandra, MongoDB en TimescaleDB geselecteerd als de meest geschikte technologieën voor Edge Computing-omgevingen. Deze keuze is gemaakt afhankelijk van hun prestaties, schaalbaarheid en geschiktheid voor toepassingen in gedistribueerde systemen. In dit onderdeel worden de redenen voor deze selectie en hun specifieke voordelen besproken.


    %\subsection{Cassandra}
    %Apache Cassandra onderscheidt zich door zijn gebruik van hash-based partitionering voor de distributie van gegevens. Deze techniek maakt het mogelijk om gegevens efficiënt te verdelen over meerdere nodes, zelfs wanneer de infrastructuur voortdurend verandert. Dit maakt Cassandra bijzonder geschikt voor dynamische Edge Computing-omgevingen, waar nodes regelmatig worden toegevoegd of verwijderd. De hoge fouttolerantie en horizontale schaalbaarheid maken Cassandra een logische keuze voor grootschalige IoT-netwerken en andere toepassingen waarbij snelheid en betrouwbaarheid cruciaal zijn.

    %Cassandra is geoptimaliseerd voor het leveren van hoge doorvoersnelheden, zelfs onder zware belasting, waardoor het bijzonder geschikt is voor toepassingen die constant grote hoeveelheden data verwerken zoals real-time analyse van sociale media of het volgen van logistieke processen. Daarnaast biedt Cassandra flexibele consistentiemodellen, waardoor gebruikers de balans kunnen bepalen tussen snelheid en gegevensintegriteit, afhankelijk van de specifieke eisen van de toepassing.

    %\subsection{MongoDB}
    %MongoDB is een flexibele NoSQL-database die sharding gebruikt om gegevens horizontaal te verdelen over meerdere servers. Deze techniek maakt MongoDB uitstekend geschikt voor Edge Computing-omgevingen, waar flexibiliteit en schaalbaarheid vereist zijn. Dankzij de ondersteuning voor zowel range-based als hash-based partitionering biedt MongoDB een optimale balans tussen prestaties en flexibiliteit.

    %MongoDB is bijzonder geschikt voor toepassingen met semi-gestructureerde data zoals documenten en JSON-objecten. Dit maakt het ideaal voor IoT-toepassingen, waarbij de datavormen sterk kunnen variëren. Bovendien biedt MongoDB sterke ondersteuning voor replicatie en automatische failover, wat zorgt voor hoge beschikbaarheid en fouttolerantie. Typische toepassingen zijn gezondheidszorg, slimme steden en e-commerce, waar veelzijdigheid en schaalbaarheid cruciaal zijn.

    %\subsection{TimescaleDB}
    %TimescaleDB, gebouwd op PostgreSQL, is ontworpen voor de opslag en verwerking van tijdreeksdata. Het gebruik van hypertables voor automatische partitionering op basis van tijd zorgt voor efficiënte opslag en snelle toegang tot data. Dit maakt TimescaleDB bij uitstek geschikt voor toepassingen waarin grote hoeveelheden tijdgebaseerde gegevens worden gegenereerd zoals sensorgegevens in IoT-omgevingen.

    %Daarnaast biedt TimescaleDB uitgebreide queryoptimalisaties en compressiemogelijkheden, waardoor het bijzonder geschikt is voor langetermijnopslag en analyse van historische data. Toepassingen zoals energiemanagementsystemen en industriële IoT-oplossingen profiteren van deze eigenschappen door real-time trends te analyseren en te voorspellen.


    \section{Conclusie}

    Uit de literatuurstudie blijkt dat Edge Computing een essentiële rol speelt in het verbeteren van de efficiëntie van gedistribueerde systemen. Door gegevensverwerking dichter bij de bron te brengen, worden latentie en bandbreedteverbruik aanzienlijk verminderd. Dit verhoogt niet alleen de snelheid, maar ook de betrouwbaarheid en veiligheid, wat van groot belang is voor real-time toepassingen zoals IoT, autonome voertuigen en industriële automatisering.

    Een belangrijk aspect binnen Edge Computing is het gebruik van data-partitioneringstechnieken. Hash-based partitionering biedt een uitgebreide oplossing voor dynamische omgevingen, waarin nodes regelmatig worden toegevoegd of verwijderd. Deze techniek verdeelt de belasting gelijkmatig en minimaliseert de herverdeling van gegevens bij wijzigingen in de infrastructuur. Range-based partitionering blijkt bijzonder effectief in toepassingen waar gegevens gelijkmatig verdeeld zijn en voorspelbare querypatronen voorkomen zoals tijdreeksanalyse. Beide technieken dragen bij aan schaalbare en efficiënte dataverwerking, waarbij de keuze tussen deze methoden afhankelijk is van de specifieke eisen van de toepassing.

    De evaluatie van databases onderstreept het belang van technologieën die optimaal aansluiten bij de vereisten van Edge Computing. PostgreSQL biedt uitgebreide ondersteuning voor gestructureerde data en geavanceerde partitioneringsopties, wat het geschikt maakt voor toepassingen waar betrouwbaarheid en gegevensintegriteit essentieel zijn. Cassandra onderscheidt zich door zijn fouttolerantie en horizontale schaalbaarheid, waardoor het een logische keuze is voor grootschalige IoT-netwerken. Redis, met zijn in-memory architectuur, levert uitzonderlijk lage latentie en is daarom bijzonder geschikt voor real-time toepassingen zoals caching en messaging. Ten slotte biedt TimescaleDB, dankzij zijn geoptimaliseerde ondersteuning voor tijdreeksdata, een krachtige oplossing voor IoT-sensoranalyse en industriële monitoring.

    De studie benadrukt daarnaast dat beveiliging en privacy een belangrijke rol spelen in Edge Computing-architecturen. Gedistribueerde databases bieden unieke voordelen zoals lokale gegevensverwerking en efficiënte synchronisatie, die bijdragen aan verbeterde gegevensbescherming zonder concessies te doen aan de prestaties. Deze eigenschappen maken ze bijzonder waardevol voor toepassingen in sectoren zoals gezondheidszorg en slimme steden.

    De inzichten uit deze literatuurstudie vormen een goede basis voor de selectie van geschikte databaseoplossingen en het ontwerp van schaalbare Edge Computing-architecturen. De bevindingen zullen dienen als richtlijn voor de verdere vergelijkende analyse en de implementatie van de proof of concept in dit onderzoek.